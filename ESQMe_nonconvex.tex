%Version 2.1 April 2023
% See section 11 of the User Manual for version history
%
%%%%%%%%%%%%%%%%%%%%%%%%%%%%%%%%%%%%%%%%%%%%%%%%%%%%%%%%%%%%%%%%%%%%%%
%%                                                                 %%
%% Please do not use \input{...} to include other tex files.       %%
%% Submit your LaTeX manuscript as one .tex document.              %%
%%                                                                 %%
%% All additional figures and files should be attached             %%
%% separately and not embedded in the \TeX\ document itself.       %%
%%                                                                 %%
%%%%%%%%%%%%%%%%%%%%%%%%%%%%%%%%%%%%%%%%%%%%%%%%%%%%%%%%%%%%%%%%%%%%%

%%\documentclass[referee,sn-basic]{sn-jnl}% referee option is meant for double line spacing

%%=======================================================%%
%% to print line numbers in the margin use lineno option %%
%%=======================================================%%

%%\documentclass[lineno,sn-basic]{sn-jnl}% Basic Springer Nature Reference Style/Chemistry Reference Style

%%======================================================%%
%% to compile with pdflatex/xelatex use pdflatex option %%
%%======================================================%%

%%\documentclass[pdflatex,sn-basic]{sn-jnl}% Basic Springer Nature Reference Style/Chemistry Reference Style


%%Note: the following reference styles support Namedate and Numbered referencing. By default the style follows the most common style. To switch between the options you can add or remove �Numbered?in the optional parenthesis.
%%The option is available for: sn-basic.bst, sn-vancouver.bst, sn-chicago.bst, sn-mathphys.bst. %

%%\documentclass[sn-nature]{sn-jnl}% Style for submissions to Nature Portfolio journals
%%\documentclass[sn-basic]{sn-jnl}% Basic Springer Nature Reference Style/Chemistry Reference Style
\documentclass[sn-mathphys,Numbered]{sn-jnl}% Math and Physical Sciences Reference Style
%%\documentclass[sn-aps]{sn-jnl}% American Physical Society (APS) Reference Style
%%\documentclass[sn-vancouver,Numbered]{sn-jnl}% Vancouver Reference Style
%%\documentclass[sn-apa]{sn-jnl}% APA Reference Style
%%\documentclass[sn-chicago]{sn-jnl}% Chicago-based Humanities Reference Style
%%\documentclass[default]{sn-jnl}% Default
%%\documentclass[default,iicol]{sn-jnl}% Default with double column layout

%%%% Standard Packages
%%<additional latex packages if required can be included here>

\usepackage{multirow,diagbox,multicol,booktabs}
\usepackage{graphicx}%
\usepackage{amsmath,amssymb,amsfonts}%
\usepackage{amsthm}%
\usepackage{mathrsfs}%
\usepackage{xcolor}%
\usepackage{textcomp}%
\usepackage{manyfoot}%
%\usepackage{refcheck}%
\usepackage{algorithm,algorithmic}%

\usepackage{latexsym,enumerate,verbatim,amsfonts,microtype}
\usepackage{color} % added by TK
\usepackage{graphicx}
\usepackage[active]{srcltx}
\usepackage{cases}
\usepackage[colorinlistoftodos,prependcaption,textsize=footnotesize]{todonotes}
\usepackage[framemethod=tikz]{mdframed}% added by TK. Create nice boxes
\mdfsetup{%
	skipbelow=4pt,
	skipabove=8pt,
	linewidth=1.25pt,
	backgroundcolor=gray!10,
	userdefinedwidth=\textwidth,
	roundcorner=10pt,
}

\allowdisplaybreaks[2]
\numberwithin{equation}{section}

% def by Ting Kei Pong
\def\cS{{\mathcal{S}}}
\def\cU{{\mathcal{U}}}
\def\cF{{\mathcal{F}}}
\def\cD{{\mathcal{D}}}
\def\cL{{\mathcal{L}}}
\def\R{{\rm I\!R}}
\def\tr{{\rm tr}}
\def\argmin{\mathop{\rm arg\,min}}
\def\Argmin{\mathop{\rm Arg\,min}}
\def\argmax{\mathop{\rm arg\,max}}
\def\Diag{{\rm Diag}}
\def\tx{{\widetilde x}}
\def\hx{{\widehat x}}
\def\conv{{\rm conv}}

\def\prox{{\rm Prox}}
\def\xfeas{x^\odot}
\def\betamin{{\rm \beta_{min}}}
\def\sigmamin{{\rm\tilde{\sigma}_{min}}}
\def\diag{{\rm diag}}
\def\d{{\rm dist}}
\def\dom{{\rm dom}\,}
\def\xorig{{x_{\rm orig}}}
%\def\xfeasss{x^\circledcirc}
%\def\xfeasss{x^\circledast}

\newcommand{\ling}[3]{{\rm lin}_{g_{#1}}(#2,#3)}

%%%%%=============================================================================%%%%
%%%%  Remarks: This template is provided to aid authors with the preparation
%%%%  of original research articles intended for submission to journals published
%%%%  by Springer Nature. The guidance has been prepared in partnership with
%%%%  production teams to conform to Springer Nature technical requirements.
%%%%  Editorial and presentation requirements differ among journal portfolios and
%%%%  research disciplines. You may find sections in this template are irrelevant
%%%%  to your work and are empowered to omit any such section if allowed by the
%%%%  journal you intend to submit to. The submission guidelines and policies
%%%%  of the journal take precedence. A detailed User Manual is available in the
%%%%  template package for technical guidance.
%%%%%=============================================================================%%%%

%\jyear{2021}%

%% as per the requirement new theorem styles can be included as shown below
%\theoremstyle{thmstyleone}%
\newtheorem{theorem}{Theorem}[section]%  meant for continuous numbers
%%\newtheorem{theorem}{Theorem}[section]% meant for sectionwise numbers
%% optional argument [theorem] produces theorem numbering sequence instead of independent numbers for Proposition
%\newtheorem{proposition}[theorem]{Proposition}%
%%\newtheorem{proposition}{Proposition}% to get separate numbers for theorem and proposition etc.
\newtheorem{definition}{Definition}[section]%
\newtheorem{lemma}{Lemma}[section]
%\newtheorem{definition}{Definition}[section]
\newtheorem{fact}{Fact}[section]
%\newtheorem{theorem}{Theorem}[section]
\newtheorem{corollary}{Corollary}[section]
\newtheorem{proposition}{Proposition}[section]
%\newtheorem{remark}{Remark}[section]
%\newtheorem{example}{Example}[section]
\newtheorem{assumption}{Assumption}[section]

%\theoremstyle{thmstyletwo}%
\newtheorem{example}{Example}[section]%
\newtheorem{remark}{Remark}[section]%

%\theoremstyle{thmstylethree}%

\raggedbottom
%%\unnumbered% uncomment this for unnumbered level heads

\begin{document}

\title[ESQM]{The extended sequential quadratic method with extrapolation}

%%=============================================================%%
%% Prefix	-> \pfx{Dr}
%% GivenName	-> \fnm{Joergen W.}
%% Particle	-> \spfx{van der} -> surname prefix
%% FamilyName	-> \sur{Ploeg}
%% Suffix	-> \sfx{IV}
%% NatureName	-> \tanm{Poet Laureate} -> Title after name
%% Degrees	-> \dgr{MSc, PhD}
%% \author*[1,2]{\pfx{Dr} \fnm{Joergen W.} \spfx{van der} \sur{Ploeg} \sfx{IV} \tanm{Poet Laureate}
%%                 \dgr{MSc, PhD}}\email{iauthor@gmail.com}
%%=============================================================%%

\author*[1]{\fnm{Yongle} \sur{Zhang}}\email{xxx.com}

\author[2]{\fnm{Ting Kei} \sur{Pong}}\email{tk.pong@polyu.edu.hk}
%\equalcont{These authors contributed equally to this work.}

\author[3]{\fnm{Siqi} \sur{Xu}}\email{xxx.com}
%\equalcont{These authors contributed equally to this work.}

\affil*[1]{\orgdiv{Department}, \orgname{Organization}, \orgaddress{\street{Street}, \city{City}, \postcode{100190}, \state{State}, \country{Country}}}

\affil[2]{\orgdiv{Department of Applied Mathematics}, \orgname{the Hong Kong Polytechnic University}, \orgaddress{\city{Hong Kong}, \country{People's Republic of China}}}

\affil[3]{\orgdiv{Department}, \orgname{Organization}, \orgaddress{\street{Street}, \city{City}, \postcode{10587}, \state{State}, \country{Country}}}

%%==================================%%
%% sample for unstructured abstract %%
%%==================================%%

\abstract{The abstract serves both as a general introduction to the topic and as a brief, non-technical summary of the main results and their implications. Authors are advised to check the author instructions for the journal they are submitting to for word limits and if structural elements like subheadings, citations, or equations are permitted.}

\keywords{keyword1, Keyword2, Keyword3, Keyword4}

%%\pacs[JEL Classification]{D8, H51}

%%\pacs[MSC Classification]{35A01, 65L10, 65L12, 65L20, 65L70}

\maketitle



\section{Introduction}

 In this paper, we consider the following difference-of-convex (DC) optimization problem with smooth inequality and simple geometric constraints:
\begin{align}\label{eq1}
\min_{x\in \mathbb {R}^n }\quad& P(x):= P_1(x) - P_2(x) \notag \\
\text{s.t.}\quad & g_i(x)\leq 0,~~ i = 1,\ldots, m,\\
& x\in C,  \notag
\end{align}
where $P_1:\mathbb{R}^n \to \R$ and $P_2:\mathbb{R}^n\to \R$ are finite-valued convex functions, $g_i:\mathbb{R}^n\to (-\infty,+\infty]$ is a smooth function, and $\nabla g_i$ is Lipschitz with continuity modulus $L_{g_i}>0$. $C\subseteq \mathbb {R}^n$ is a nonempty compact convex set, the feasible set $\mathcal{G} = C\cap \mathscr{F}$ is nonempty with $\mathscr{F}=\{x\in \mathbb{R}^n : g_i(x)\leq 0, ~i = 1,\dots,m\}$.

\section{Notation and preliminaries}\label{sec2}

In this paper, we use $\mathbb{R}$ to denote the set of real numbers, use $\mathbb{R}_+$ to denote the set of nonnegative real numbers and use $\mathbb{Z}$ to denote the set of integer numbers. We also use $\mathbb{R}^n$ to denote the Euclidean space of dimension $n$, and use $\mathbb{R}_+^n$ to denote the nonnegative orthant Euclidean space of dimension $n$. For a vector $x\in \mathbb{R}^n$, $\|x\|$ stands for the Euclidean norm of $x$. For two vectors $x$ and $y\in \mathbb{R}^n$, $\langle x,y \rangle$ stands for the inner product of $x$ and $y$.

For extended-real-valued function $f:\mathbb{R}^n \to (-\infty,+\infty]$, we say $f$ is a proper function if $\dom f := \{ x: f(x)<\infty \}\neq \emptyset$. The proper function $f$ is considered to be closed if it is lower semicontinuous. We use $x^k\overset{f}{\to} x$ to stand for $x^k\to x$ and $f(x^k)\to f(x)$. For a proper closed function $f$, the regular subdifferential of $f$ at $w\in \dom f$ is given by
$$
\widehat{\partial} f(w):= \left\{ \xi\in\mathbb{R}^n :\liminf_{v\to w,v\neq w} \frac{f(v) - f(w) - \langle \xi, v - w \rangle }{\| v - w\|}\geq 0 \right\}.
$$
The (limiting) subdifferential of $f$ at $w\in \dom f$ is given by
\begin{equation*}
\partial f(w):= \left\{ \xi\in\mathbb{R}^n:\exists w^{k}\overset{f}{\to} w, \xi^{k}\to\xi \text{ with } \xi^{k} \in \widehat{\partial} f(w^{k}) \text{ for each } k\right\}.
\end{equation*}
In general, if $ x\notin \dom f$, then $\partial f(x)=\widehat{\partial}f(x) = \emptyset,$ and $\dom \partial f := \{x:\partial f(x)\neq \emptyset\}$. The above subdifferential of $f$ is consistent with the classical subdifferential of $f$ when $f$ is a convex function; see, for example, \cite[proposition 8.12]{rock97a} and \cite[proposition 8.8]{rock97a}. If $f$ is a convex function, then the subdifferential of $f$ at $w\in \dom f$ is given by
$$
\partial f(w)=\left\{ \xi\in\mathbb{R}^n:\langle \xi, v - w \rangle\leq f(v) - f(w) ~~ \forall v\in \mathbb{R}^n \right\}.
$$
For a nonempty closed set $C\subseteq \mathbb{R}^n$, the indicator function $\delta_C$ is defined by
\begin{equation*}
\delta_C(x) = \left\{
\begin{array}{lr}
 0~~ &x\in C,
 \\  \infty ~~ &x\notin C.
 \end{array}
 \right.
\end{equation*}
The normal cone of $C$ at $x\in C$ is defined by $$ \mathcal{N}_C(x) :=\partial \delta_C(x)=\{\xi \in \mathbb{R}^n:\langle \xi, y - x \rangle \leq 0 ~~\forall y\in C\}.$$
The distance from a point $x$ to $C$ is denoted by $\d(x, C)$. The convex hull of $C$ is denoted by $\conv ~C$.

We next recall a constraint qualification for \eqref{eq1} (which first appeared in \cite{Ausleder13}), and the first-order optimality conditions for \eqref{eq1}.
%
%\begin{definition}[{{\bf MFCQ}}]\label{MFCQ}
% Mangassarian-Fromovitz constraint qualification holds at $x\in C\cap \mathscr{F}$ if the following statement holds:
%\[
%\left. \begin{gathered}
%-\sum_{i=1}^{m}\lambda_i\nabla g_i(x)\in \mathcal{N}_C(x)
%\\	\lambda_i g_i(x)=0, ~ i=1,\dots,m
%\\	\lambda_i \geq 0, ~ i=1,\dots,m
%\end{gathered} \right\}
%\implies \lambda_i=0, ~ i=1,\dots,m. 	\]
%\end{definition}
%Note that if $g_i$ in \eqref{eq1} is a convex function and the Slater condition holds, i.e., $\{x \in C: g_i(x) < 0, ~ i = 1, \ldots, m\} \neq \emptyset $, then the MFCQ holds at any point in $G$.

\begin{definition}[{{\bf RCQ}}]\label{RCQ}
We say that the Robinson constraint qualification holds at $x\in\R^n$ for \eqref{eq1} if the following statement holds:
$$ RCQ(x):~\exists y\in C, \text{ such that } g_i(x) + \langle\nabla g_i(x), y-x\rangle < 0~~ \forall i = 1,2,\cdots m.$$
\end{definition}

%Note that if $RCQ(x)$ holds at $x\in C\cap \mathscr{F}$, then MFCQ holds at $x\in C\cap \mathscr{F}$.

\begin{definition}[{{\bf Critical point}}]\label{Stationary}
For \eqref{eq1}, we say that $x$ is a critical point of \eqref{eq1} if $x\in C$ and there exists $\lambda=(\lambda_1, \lambda_2, \dots, \lambda_m)\in \mathbb{R}_+^m$ such that $(x, \lambda)$ satisfies the following conditions:
\begin{enumerate}[{\rm (i)}]
    \item $ g_i(x)\leq 0 ~~\forall i=1,\dots,m,$
    \item $ \lambda_i g_i(x)=0 ~~\forall i=1,\dots,m,$
    \item $0\in\partial P_1(x) - \partial P_2(x) + \sum_{i=1}^{m}\lambda_i\nabla g_i(x) + \mathcal{N}_C(x).$
\end{enumerate}	
\end{definition}

If RCQ holds at every point in $\mathcal{G}$, by using similar arguments as in \cite[Section 2]{yu21}, one can show that any local minimizer of \eqref{eq1} is a critical point of \eqref{eq1}.

Next, we recall the notation of Kurdyka-{\L}ojasiewicz (KL) property and Kurdyka-{\L}ojasiewicz (KL) exponent. The KL property plays an important role in the convergence rate analysis of first order methods; see, for example, \cite{bolte14,attouch13,beck09,bolte07,attouch10}.
\begin{definition}[{{\bf Kurdyka-{\L}ojasiewicz (KL) property and exponent}}]\label{KLd}
A proper closed function $f$ is said to satisfy the KL property at $\bar{x}\in \text{dom} \, \partial f$ if there exist $r\in (0,\infty]$, a neighborhood $U$ of $\bar{x}$, and a continuous concave function $\phi:[0,r)\to \mathbb{R}_+$, $\phi(0)=0$ such that:
\begin{enumerate}[{\rm (i)}]
    \item $\phi$ is continuously differentiable on $(0,r)$ with $\phi'>0$;
    \item For all $x\in U$, $f(\bar{x})< f(x) < f(\bar{x}) + r$, it holds that
        \begin{align}\label{eq21}
        \phi'(f(x) - f(\bar{x}))\d(0,\partial f(x))\geq 1.
        \end{align}
\end{enumerate}
If $f$ satisfies the KL property at $\bar{x}\in \dom\partial f$, and $\phi$ in \eqref{eq21} can be chosen as $\phi(\varsigma)= \rho \varsigma^{1-\alpha}$ for some $\rho>0$ and $\alpha\in[0,1)$, then we say that $f$ satisfies the KL property with exponent $\alpha$ at $\bar{x}$.

A proper closed function $f$ satisfies KL property at every point in dom\,$\partial f$ is called KL function. Moreover, a proper closed function $f$ satisfies KL property with exponent $\alpha\in[0,1)$ at every point in dom\,$\partial f$ is called KL function with exponent $\alpha$.
\end{definition}

Based on the above definition, we next recall the following Lemma, which was established in \cite[Lemma~3.10]{yu21}.
%\begin{lemma}[{{\bf Uniformized KL property}}]\label{uKL}
%Suppose that $f$ is a proper closed function and $\mathcal{V}$ is a compact set. If $f$ is continuous at $\mathcal{V}$ and satisfies the KL property at every point in $\mathcal{V}$, then there exists $\epsilon>0, \eta>0, \phi\in \varTheta_\eta$ such that
%$$\phi'(f(x)-f(\bar{x}))\text{dist}(0,\partial f(x))\geq 1,$$
%holds for all $\bar{x}\in \mathcal{V}$, and any $x$ satisfies dist$(x, \mathcal{V})<\epsilon$ and $f(\bar{x})<f(x)<f(\bar{x})+\eta$.
%\end{lemma}

\begin{lemma}\label{KLinequ}
Let $f:\R^n\rightarrow (-\infty,+\infty]$ be a level-bounded proper closed convex function with $\Lambda:= \Argmin f\not=\emptyset$. Let $\underline{f}:=\inf f$. Suppose that $f$ satisfies the KL property at each point in $\Lambda$ with exponent $\alpha\in[0,1)$. Then there exist $\epsilon >0$, $r_0>0$, and $c_0>0$ such that
\[
\d(x, \Lambda)\leq c_0(f(x) - \underline{f})^{1-\alpha}
\]
for any $x\in\dom \partial f$ satisfying $\d(x, \Lambda)\leq \epsilon$ and $\underline{f}\leq f(x) \leq \underline{f} + r_0$.
\end{lemma}

The following lemma is a special case of Robinson \cite{Rob75} concerning error bounds for convex functions.

\begin{lemma}\label{RobEB}
Let $g:\R^n\to \R^m$ be a convex function. Let $\Omega := \{x\in \mathbb{X}:\; 0 \in g(x) + \R^m_+\}$ and suppose there exist $x^s\in \Omega$ and $\delta_0 > 0$ such that $B(0,\delta_0)\subseteq g(x^s) + \R^m_+$.
Then
\[
\d(x,\Omega)\leq \frac{\|x - x^s\|}{\delta_0}\d(0, g(x) + \R^m_+)~~  \forall x\in \mathbb{X}.
\]
\end{lemma}

\section{Algorithm and convergence analysis}\label{sec3}

From \cite{wen17}, for each $i$, we see that $g_i$ (whose gradient is Lipschitz continuous) can be written as $g_i = g_i^1 - g_i^2$, where $g_i^1$ and $g_i^2$ are two convex functions with Lipschitz continuous gradients. We first present a remark concerning the Lipschitz continuity moduli of $\nabla g_i^1$ and $\nabla g_i^2$.

\begin{remark}\label{Remarkg}
We denote a Lipschitz continuity modulus of $\nabla g_i^1$ by $L_{g_i} > 0$ and a Lipschitz continuity modulus of $\nabla g_i^2$ by $\ell_{g_i} > 0$, by taking a larger $L_{g_i}$ if necessary, we may assume without loss of generality that $L_{g_i} \geq \ell_{g_i}$. Then one can show that $\nabla g_i$ is Lipschitz continuous with a modulus $L_{g_i}$. We let $L_g := \max\{L_{g_i}: i=1,\dots,m\}$ and $\ell_g = \max\{\ell_{g_i}: i=1,\dots,m\}$ for brevity.
\end{remark}

The extended sequential quadratic method with extrapolation (ESQM$_{\text{e}}$) we study in this paper is presented in Algorithm \ref{alg:Framwork} below. Here and throughout, for notational simplicity, for each $u$, $w\in \R^n$, we define
\begin{equation}\label{ling}
\ling{0}{u}{v}\equiv 0 \ \ {\rm and}\ \ \ling{i}{u}{w} := g_i(w) + \langle\nabla g_i(w), u - w\rangle \ \ \ \forall i = 1,\ldots,m.
\end{equation}
\begin{algorithm}
\caption{ESQM$_{\text{e}}$ for solving problem \eqref{eq1}}\label{alg:Framwork}
\begin{algorithmic}
\STATE
\begin{description}
  \item[\bf Step 0.] Choose $x^{-1}=x^0\in C, \{\beta_k\}\subseteq\left[0,\sqrt{\frac{L_g}{L_g + \ell_g}}~\right]$, $\theta_0>0$, $d>0$, where $L_g = \max\{L_{g_i}, ~ i=1,\dots,m\}$ and $\ell_g = \max\{\ell_{g_i}, ~ i=1,\dots,m\}$.
  \item[\bf Step 1.] Set
    \begin{equation}\label{defyk}
      y^k = x^k + \beta_k(x^k - x^{k-1}).
    \end{equation}
  \item[\bf Step 2.] Take any $\xi^k\in \partial P_2(x^{k})$ and compute
    \begin{align}\label{eq2}
    (x^{k+1},s^{k+1})=\Argmin\limits_{(x,s)\in \mathbb{R}^{n+1}}\quad &P_1(x) - \langle \xi^k, x \rangle
	+ \theta_k s + \frac{\theta_k L_g}{2}\| x - y^k \|^2   \notag
	\\ \text{s.t.} \quad &\ling{i}{x}{y^k} \leq s ~~\forall i=1,\dots,m,
	\\& (x,s)\in C\times \mathbb{R_+},  \notag
	\end{align}
where ${\rm lin}_{g_i}$ is defined in \eqref{ling}.
  \item[\bf Step 3.]  If $\ling{i}{x^{k+1}}{y^k}\leq 0$ for all $i$, then $\theta_{k+1}=\theta_k$; otherwise $\theta_{k+1}=\theta_k+d$. Update $k\leftarrow k+1$ and go to step 1.
\end{description}
\end{algorithmic}
\end{algorithm}

We next present some useful facts concerning the subproblem \eqref{eq2}.

\begin{lemma}\label{subproremarks}
Suppose that $x^{k-1}, x^k\in C$ are generated at the beginning of the $k$-th iteration of Algorithm \ref{alg:Framwork} for some $k\geq 0$. Then the following statements hold:
\begin{enumerate}[{\rm (i)}]
    \item $s^{k+1} = \max_{i = 1,\cdots,m}[\ling{i}{x^{k+1}}{y^k}]_+$.
    \item Problem \eqref{eq2} has a unique solution.
    \item Let $g_0:\equiv 0$. Then, for each $k\geq 0$, $x^{k+1}$ is a component of the minimizer of \eqref{eq2} if and only if there exist $\lambda_i^k \geq 0$ for all $i\in I_k(x^{k+1})$ such that $\sum_{i\in I_k(x^{k+1})}\lambda_i^k = 1$ and
        \begin{equation}\label{KKT2}
        0\in \partial P_1(x^{k+1}) - \xi^k + \theta_k\sum_{i\in I_k(x^{k+1})} \lambda_i^k\nabla g_i(y^k) + \theta_kL_g(x^{k+1} - y^k) + \mathcal{N}_C(x^{k+1}),
        \end{equation}
        where
        \begin{equation}\label{defiIk}
        I_k(x): = \left\{s\in\{0,1,\cdots,m\}: \ling{s}{x}{y^k} = \max_{i = 0, 1,\cdots,m}\ling{i}{x}{y^k}\right\}.
        \end{equation}
\end{enumerate}
\end{lemma}

\begin{proof}
It is not hard to show that (i) and (ii) hold. We omit the proof for brevity.

Now, we prove (iii). Combined $g_0:\equiv 0$ with (i), we have that $x^{k+1}$ in \eqref{eq2} satisfies
\begin{equation}\label{subproblem2}
x^{k+1} = \Argmin\limits_{x\in C}~ P_1(x) - \langle \xi^k, x \rangle + \theta_k \max_{i = 0, 1,\cdots,m}\{\ling{i}{x}{y^k}\} + \frac{\theta_k L_g}{2}\| x - y^k \|^2.
\end{equation}
Then, from \cite[Theorem 23.8]{Ro70}, we have that $x^{k+1}$ is a minimizer of the convex problem \eqref{subproblem2} if and only if
\begin{align*}
0&\!\in\! \partial P_1(x^{k+1}) \!-\! \xi^k \!+\! \theta_k\partial\!\left(\max_{i = 0, 1,\cdots,m}\{\ling{i}{\cdot}{y^k}\}\right)\!(x^{k+1}) \!+\! \theta_kL_g(x^{k+1} - y^k) \!+\! \mathcal{N}_C(x^{k+1})\\
& \!\!\overset{\rm(a)}=\! \partial P_1(x^{k+1}) - \xi^k + \theta_k\conv\{\nabla g_i(y^k): i\in I_k(x^{k+1})\} + \theta_kL_g(x^{k+1} - y^k) + \mathcal{N}_C(x^{k+1}),
\end{align*}
where $I_k(\cdot)$ defined in \eqref{defiIk}, and (a) follows from \cite[Exercise~8.31]{rock97a}.

That is, for each $k\geq 0$, there exist $\lambda_i^k \geq 0$ for all $i\in I_k(x^{k+1})$ such that $\sum_{i\in I_k(x^{k+1})}\lambda_i^k = 1$ and
\begin{equation*}
0\in \partial P_1(x^{k+1}) - \xi^k + \theta_k\sum_{i\in I_k(x^{k+1})} \lambda_i^k\nabla g_i(y^k) + \theta_kL_g(x^{k+1} - y^k) + \mathcal{N}_C(x^{k+1}).
\end{equation*}
This completes the proof.
\end{proof}


\section{Convergence properties}
\subsection{Convergence analysis for ESQM$_{\rm e}$ in Algorithm~\ref{alg:Framwork}}

\begin{theorem}\label{suffdec}
Consider \eqref{eq1}. Let $\{(x^k,y^k,s^k,\theta_k)\}$ be the sequence generated by Algorithm \ref{alg:Framwork}. Then the following statements hold:
\begin{enumerate}[{\rm (i)}]
    \item The sequence $\{x^k\}$ belongs to $C$ and is bounded.
    \item Let $\bar{m} = \inf\{P(x):x\in C\}$. Then for any $k\geq 0$,
        $$
        Q(x^{k+1},x^{k},y^{k},\theta_{k+1}) \leq Q(x^k,x^{k-1},y^{k-1},\theta_k) - \left(1 - \frac{L_g + \ell_g}{L_g}\beta_k^2\right)\frac{L_g}{2}\| x^{k} - x^{k-1}\|^2,
        $$
    where 
    \[
    Q(x,y,z,\theta)\!:=\!\frac{1}{\theta}\!\left(\!P(x) \!-\! \bar{m} \!+\! \theta\!\!\max_{i = 1,\cdots,m}\left[\ling{i}{x}{z}\right]_+\!+\! \frac{\theta L_g}{2}\| x \!-\! y \|^2 \!+\! \frac{\theta L_g}{2}\| x \!-\! z \|^2 \!\right)\!.
    \]
    \item $\sum_{k = 1}^{\infty} \frac{L_g - (L_g + \ell_g)\beta_k^2}{2} \| x^k - x^{k-1}\|^2 < \infty$. Moreover, if $\bar{\beta}: = \sup_k\beta_k < \sqrt{\frac{L_g}{L_g + \ell_g}}$, then $\lim_{k \rightarrow\infty}\|x^k - x^{k-1}\| = 0$ and  $\lim_{k \rightarrow\infty}\|x^k - y^k\| = 0$.
\end{enumerate}
\end{theorem}

\begin{proof}
(i): Since $\{x^k\}\subseteq C$ according to \eqref{eq2} and $C$ is a nonempty compact convex set, $\{x^k\}$ is bounded.

(ii): Using the definition of $(x^{k+1},s^{k+1})$ in \eqref{eq2} and the strong convexity of the objective in the minimization problem \eqref{eq2}, and noting that $(x^k,\max_{i = 1,\cdots,m}\left[\ling{i}{x^{k}}{y^k}\right]_+)$ is feasible for \eqref{eq2}, we have that
\begin{equation}
\begin{split}\label{eq6}
& P_1(x^{k+1}) \!-\! \langle \xi^k,x^{k+1}-x^k\rangle + \theta_k s^{k+1} + \frac{\theta_k L_g}{2}\| x^{k+1} - y^k  \|^2 \\
&\leq P_1(x^{k}) \!+\! \theta_k \max_{i = 1,\cdots,m}\left[\ling{i}{x^{k}}{y^k}\right]_+ \!+\! \frac{\theta_k L_g}{2}\| x^{k}-y^k \|^2 \!-\! \frac{\theta_k L_g}{2}\| x^{k+1}-x^k \|^2.
\end{split}
\end{equation}
Meanwhile, from Remark~\ref{Remarkg} and the definition of ${\rm lin}_{g_i}$ in \eqref{ling}, we see that
\begin{align}\label{gnonconvex}
&\!\! \max_{i = 1,\cdots,m}\left[\ling{i}{x^{k}}{y^k}\right]_+ \notag\\
&\!\!= \max_{i = 1,\cdots,m}\left[g^1_i(y^k) + \langle\nabla g^1_i(y^k), x^k - y^k \rangle - g^2_i(y^k) - \langle\nabla g^2_i(y^k), x^k - y^k \rangle\right]_+ \notag\\
&\!\!\overset{\rm(a)}{\leq}\!\! \max_{i = 1,\cdots,m}\left[g^1_i(x^{k}) \!-\! g^2_i(x^{k}) \!+\! \frac{\ell_{g_i}}{2}\|x^k \!-\! y^{k}\|^2\right]_+ \!\!\!\!=\!\!\! \max_{i = 1,\cdots,m}\left[g_i(x^{k}) \!+\! \frac{\ell_{g_i}}{2}\|x^k \!-\! y^{k}\|^2\right]_+ \\
&\!\! \overset{\rm(b)}{\leq} \max_{i = 1,\cdots,m}\left[\ling{i}{x^{k}}{y^{k-1}}+ \frac{L_{g_i}}{2}\|x^k - y^{k-1}\|^2 + \frac{\ell_{g_i}}{2}\|x^k - y^{k}\|^2\right]_+ \notag\\
&\!\!\overset{\rm(c)}{\leq} \max_{i = 1,\cdots,m}\left[\ling{i}{x^{k}}{y^{k-1}}\right]_+ + \frac{L_g}{2}\|x^k - y^{k-1}\|^2 + \frac{\ell_{g}}{2}\|x^k - y^{k}\|^2, \notag
\end{align}
where (a) holds because of the convexity of $\nabla g^1_i$ and the Lipschitz continuity of $\nabla g^2_i$, (b) follows from the Lipschitz continuity of $\nabla g_i$, and (c) holds because $L_g=\max\{L_{g_i}:\;i=1,\dots,m\}$ and $\ell_g=\max\{\ell_{g_i}:\;i=1,\dots,m\}$.

Then, we obtain that
\begin{align*}
&P(x^{k+1}) = P_1(x^{k+1}) - P_2(x^{k+1})\overset{\rm(a)}{\leq} P_1(x^{k+1}) - \langle\xi^k, x^{k+1} - x^k\rangle - P_2(x^{k})\\
& = P_1(x^{k+1}) - \langle \xi^k, x^{k+1} - x^k\rangle + \frac{\theta_k L_g}{2}\| x^{k+1} - y^k \|^2 - \frac{\theta_k L_g}{2}\| x^{k+1} - y^k\|^2 - P_2(x^{k})\\
&\overset{\rm(b)}{\leq} P_1(x^{k}) + \frac{\theta_k L_g}{2}\| x^{k}-y^k \|^2 - \theta_k s^{k+1} + \theta_k \max_{i = 1,\cdots,m}\left[\ling{i}{x^{k}}{y^{k}}\right]_+ \\
&~~~~~~ - \frac{\theta_k L_g}{2}\| x^{k+1} - x^k\|^2 - \frac{\theta_k L_g}{2}\| x^{k+1} - y^k\|^2 - P_2(x^{k}) \\
&\overset{\rm(c)}{\leq} P(x^{k}) + \theta_k \left(\max_{i = 1,\cdots,m}\left[\ling{i}{x^{k}}{y^{k-1}}\right]_+
 + \frac{L_g}{2}\| x^k - y^{k-1}\|^2 + \frac{\ell_{g}}{2}\|x^k - y^{k}\|^2\right)\\
  & ~~~~~  + \frac{\theta_k L_g}{2}\| x^{k} - y^k\|^2 - \theta_k s^{k+1} - \frac{\theta_k L_g}{2}\| x^{k+1} - x^k\|^2 - \frac{\theta_k L_g}{2}\| x^{k+1} - y^k\|^2,
\end{align*}
where (a) holds because $P_2$ is convex and $\xi^k\in \partial P_2(x^k)$, (b) holds thanks to \eqref{eq6}, and (c) holds because of \eqref{gnonconvex}.

Rearranging terms in the above display and noting that $y^k - x^k = \beta_k(x^k - x^{k - 1})$ (thanks to the definition of $y^k$ in \eqref{defyk}), we have that
\begin{align}\label{eq8}
&P(x^{k+1}) + \theta_k s^{k+1} + \frac{\theta_k L_g}{2}\| x^{k+1} - x^k\|^2 + \frac{\theta_k L_g}{2}\| x^{k+1} - y^k\|^2 \notag\\
&\leq P(x^{k}) + \theta_k\max_{i = 1,\cdots,m}\left[\ling{i}{x^{k}}{y^{k-1}}\right]_+ + \frac{\theta_k L_g}{2}\| x^k - y^{k-1}\|^2 \\
&~~~~ + \frac{\theta_k (L_g + \ell_g)}{2}\beta_k^2\| x^{k} - x^{k-1}\|^2 \notag\\
&= P(x^{k}) + \theta_k\max_{i = 1,\cdots,m}[\ling{i}{x^{k}}{y^{k-1}}]_+ + \frac{\theta_k L_g}{2}\| x^{k} - x^{k-1}\|^2 \notag\\
&~~~~~~ + \frac{\theta_k L_g}{2}\| x^k - y^{k-1}\|^2 - \left(1 - \frac{L_g + \ell_g}{L_g}\beta_k^2\right)\frac{\theta_k L_g}{2}\| x^{k} - x^{k-1}\|^2. \notag
\end{align}

Since $P$ is continuous and $C$ is a compact set, it implies that $\bar{m} = \inf\{P(x):x\in C\} > -\infty$. Then we see from the definition of $Q$ and the observation $s^{k+1} = \max_{i = 1,\cdots,m}\left[\ling{i}{x^{k+1}}{y^{k}}\right]_+$ (thanks to Lemma~\ref{subproremarks}(i)) that
\begin{align}\label{eq81}
&Q(x^{k+1},x^{k},y^{k},\theta_{k+1}) \notag\\
&= \frac{P(x^{k+1}) - \bar{m}}{\theta_{k+1}} + s^{k+1} + \frac{L_g}{2}\| x^{k+1} - x^{k} \|^2 + \frac{L_g}{2}\| x^{k+1} - y^{k} \|^2 \notag\\
&\overset{\rm(a)}{\leq} \frac{P(x^{k+1}) - \bar{m}}{\theta_k} + s^{k+1} + \frac{L_g}{2}\| x^{k+1} - x^{k} \|^2 + \frac{L_g}{2}\| x^{k+1} - y^{k} \|^2 \notag\\
&\overset{\rm(b)}{\leq} \frac{1}{\theta_k}\left( P(x^{k}) - \bar{m} + \theta_k \max_{i = 1,\cdots,m}[\ling{i}{x^{k}}{y^{k-1}}]_+ + \frac{\theta_k L_g}{2}\| x^{k} - x^{k-1}\|^2\right. \notag\\
&~~~~\left. + \frac{\theta_k L_g}{2}\|x^k - y^{k-1}\|^2- \left(1 - \frac{L_g + \ell_g}{L_g}\beta_k^2\right)\frac{\theta_k L_g}{2}\| x^{k} - x^{k-1}\|^2\right) \notag\\
&\overset{\rm(c)}{=} \frac{1}{\theta_k}\left( P(x^{k}) -\bar{m}\right) + \max_{i = 1,\cdots,m}\left[\ling{i}{x^{k}}{y^{k-1}}\right]_+ \notag\\
&~~~~ + \frac{L_g}{2}\| x^{k} - x^{k-1}\|^2 + \frac{L_g}{2}\|x^k - y^{k-1}\|^2- \left(1 - \frac{L_g + \ell_g}{L_g}\beta_k^2\right)\frac{L_g}{2}\| x^{k} - x^{k-1}\|^2 \notag\\
& = Q(x^k,x^{k-1},y^{k-1},\theta_k) - \left(1 - \frac{L_g + \ell_g}{L_g}\beta_k^2\right)\frac{L_g}{2}\| x^{k} - x^{k-1}\|^2,
\end{align}
where (a) holds because the definition of $\bar{m}$ and the fact that $\{\theta_k^{-1}\}$ is nonincreasing, which implies that $\theta_{k+1}^{-1}\leq \theta_k^{-1}$; (b) follows from \eqref{eq8} and the fact that $\frac{1}{\theta_k} > 0$, and (c) holds because $x^{k}\in C$.

(iii): Observe that, for any $k\geq 1$,
\begin{equation*}
\begin{split}
&Q(x^{k+1},x^{k},y^k,\theta_{k+1})  \\
&= \frac{P(x^{k+1}) - \bar{m}}{\theta_{k+1}} \!+\! \max_{i = 1,\cdots,m}[\ling{i}{x^{k+1}}{y^k}]_+ \!+\! \frac{L_g}{2}\| x^{k+1} - x^{k} \|^2 \!+\! \frac{L_g}{2}\| x^{k+1}-y^{k} \|^2 \!\geq\! 0.
\end{split}
\end{equation*}
Combining the above display with \eqref{eq81}, we have
\begin{align*}
&\sum_{k=1}^{\infty}\left(1 - \frac{L_g + \ell_g}{L_g}\beta_k^2\right)\frac{L_g}{2}\| x^{k} - x^{k-1} \|^2 \\
&\leq Q(x^1,x^{0},y^{0},\theta_1) - \liminf_{k\to \infty} Q(x^{k+1},x^{k},y^k,\theta_{k+1})\leq Q(x^1,x^{0},y^{0},\theta_1) <\infty .
\end{align*}
Moreover, if $\bar{\beta}: = \sup_k\beta_k < \sqrt{\frac{L_g}{L_g + \ell_g}}$, then we deduce from the above display that
$$\lim_{k\to \infty}\| x^k - x^{k-1}\|=0.$$
Combining this with the definition of $y^k$ in \eqref{defyk}, we can obtain further that
$
\lim_{k\to \infty}\| y^k - x^k\| = \lim_{k\to \infty}\beta_k\| x^k - x^{k-1}\| =0$.
\end{proof}

Next, we need to make use of the following assumption, which was first introduced in \cite[Assumption~A1]{Ausleder13}.

\begin{assumption}\label{A1}
$RCQ(x)$ holds at every point $x\in C\cap \mathcal{F}$, and for every $x\in C\setminus \mathcal{F}$, there cannot exist $u_i$, $i\in I(x)$, such that
\begin{equation}\label{A11}
u_i\geq 0 ~~ \forall i\in I(x), ~~ \sum_{i\in I(x)}u_i=1, ~~ \left\langle\sum_{i\in I(x)}u_i\nabla g_i(x), z - x\right\rangle\geq0 ~~\forall z\in C.
\end{equation}
where $I(x) := \Big\{ s\in\{1, 2, \dots, m\}: g_s(x) = \max_{i = 1, 2, \dots, m} \{ g_i(x), 0\} \Big\}$.\footnote{Note that, from the definition of $I_k(\cdot)$ in \eqref{defiIk}, one can see that the difference between $I_k(\cdot)$ and $I(\cdot)$ is that $I_k(\cdot)\subseteq\{0, 1,\cdots,m\}$, while $I(\cdot)\subseteq\{1,\cdots,m\}$.}
\end{assumption}
\begin{remark}\label{remarkRCQ}
From the definition of $RCQ(x)$, we have that if Assumption~\ref{A1} holds, then for any $x\in C$, there cannot exist $u_i$, $i\in I(x)$, such that \eqref{A11} holds. Moreover, if $RCQ(x)$ holds at every point $x\in C$, then Assumption~\ref{A1} holds.
\end{remark}

Using this Assumption and Theorem~\ref{suffdec}, we will prove in the next theorem that the sequence $\{\theta_k\}$ in Algorithm~\ref{alg:Framwork} is bounded.

\begin{theorem}\label{alpha}
Consider \eqref{eq1} and suppose that Assumption \ref{A1} holds and $\bar{\beta}: = \sup_k\beta_k < \sqrt{\frac{L_g}{L_g + \ell_g}}$. Let $\{(x^k,y^k,\theta_k)\}$ be the sequence generated by Algorithm \ref{alg:Framwork}, $A := \{k\in \mathbb{N}:\theta_{k+1}>\theta_k\}$, and let $|A|$ denote the number of elements in A. Then $|A|$ is finite, i.e., there exists $N_0\in \mathbb{N}$, such that $\theta_k \equiv \theta_{N_0}$, whenever $k\geq N_0$. Moreover, $s^{k+1} = 0$ whenever $k\geq N_0$.
\end{theorem}

\begin{proof}
By the definition of A, $|A|$ is either finite or infinite. If $\Upsilon(\text{A}) = \infty$, by the definition of $\theta_k$ in Step~3, we have that $\lim_{k\to \infty}\theta_k = \infty$ and $\lim_{k\to \infty}\theta_k^{-1}=0$.
	
We first claim that for each $i$, there exists $n_i\in \mathbb{Z}$ such that, for all $k\geq n_i$,
\begin{equation}
 g_i(y^k) + \langle \nabla g_i(y^k),x^{k+1} - y^k \rangle\leq 0.
\end{equation}
If not, then there exists $i_0\in\{1,\dots, m\}$, and subsequences $\{x^{k_j}\}$ and $\{y^{k_j}\}$, such that
\begin{equation}\label{ifnot}
 g_{i_0}(y^{k_j}) + \langle \nabla g_{i_0}(y^{k_j}),x^{{k_j}+1} - y^{k_j} \rangle > 0.
\end{equation}
Then, by the definition of $I_k(\cdot)$ in \eqref{defiIk} and \eqref{ifnot}, we have that for all $i\in I_{k_j}(x^{k_j+1})$,
\begin{equation*}
 g_{i}(y^{k_j}) + \langle \nabla g_{i}(y^{k_j}),x^{{k_j}+1} - y^{k_j} \rangle > 0.
\end{equation*}

In view of the infiniteness of $\{(x^{k_j+1}, y^{k_j})\}$ and the finiteness of $\left\{I_{k_j}(x^{k_j+1})\right\}$ (since $I_{k_j}(x^{k_j+1})\subseteq \{0, 1, \dots, m\}$ for all $j$), passing to a further subsequence if necessary, there exists a nonempty subset $I_0\subseteq \{0, 1,\dots,m\}$ with $0\notin I_0$ such that
$I_0 \equiv I_{k_j}(x^{k_j+1})$ for all $j$. That is, for all $i\in I_0$,
\begin{equation}\label{eq10}
\ling{i}{x^{k_j+1}}{y^{k_j}} = \max_{s = 0, 1, \dots, m} \left\{ \ling{s}{x^{k_j+1}}{y^{k_j}}\right\} > 0~~ \forall j.
\end{equation}

From Lemma~\ref{subproremarks}(iii), we have that for each $k_j$, there exist $\lambda_i^{k_j} \geq 0$ for each $i\in I_{k_j}(x^{k_j + 1}) (\equiv I_0)$, such that $\sum_{i\in I_0}\lambda_i^{k_j} = 1$ and
\begin{align}\label{eq11}
0\in \theta_{k_j}^{-1}(\partial P_1(x^{k_j+1}) - \xi^{k_j}) + L_g(x^{k_j+1} - y^{k_j}) \!+\! \sum_{i\in I_0} \lambda_i^{k_j} \nabla g_i(y^{k_j}) \!+\! \mathcal{N}_C(x^{k_j+1}).
\end{align}

Since the sequences $\{x^k\}$ and $\{\lambda_i^{k_j}: i \in I_0\}$ are bounded, by passing to a further subsequence if necessary, we assume $\{x^{k_j}\}$ is a convergent subsequence such that $\lim_{j \to \infty} x^{k_j} = x^{*}$ and $\lim_{j \to \infty}\lambda_i^{k_j}= \bar{\lambda}_i$, for all $i\in I_0$. Then $x^*\in C$, $\bar{\lambda}_i \ge 0$ (for each $i\in I_0$), $\sum_{i\in I_0} \bar{\lambda}_i = 1$ and $I_0 \subseteq \left\{i \in\{0,1,\cdots,m\}: g_i(x^*) = \max_{i=0,1,\cdots,m} g_i(x^*)\right\}$ (thanks to $I_0 = I_{k_j}(x^{k_j+1})$ for all $k_j$). Since $0\notin I_0$, we see that
$$I_0 \subseteq I(x^*) :=\left\{i \in\{1,2,\cdots,m\}: g_i(x^*) = \max_{i=1,\cdots,m} \{g_i(x^*), 0\}\right\}.$$

Passing to the limit in \eqref{eq11}, and noting that $\lim_{j \to \infty} \theta_{k_j}^{-1} = 0$, $\lim_{j \to \infty}\| x^{k_j+1} - y^{k_j} \|=0$ (thanks to Theorem~\ref{suffdec}(iii)), $\{\partial P_1(x^{k_j+1})\}$ and $\{\xi^{k_j}\}$ are uniformly bounded (thanks to $P_1$, $P_2$ are convex, $C$ is bounded and \cite[Theorem~2.6]{Tu98}), we have that
$$0\in \sum_{i\in I_0}\bar{\lambda}_i \nabla g_i(x^*) + \mathcal{N}_C(x^*),$$
which implies that
\begin{align}\label{eq12}
\left\langle \sum_{i\in I_0}\bar{\lambda}_i \nabla g_i(x^*), x-x^* \right\rangle \geq 0 ~~ \forall x\in C.
\end{align}
Since $I_0\subseteq I(x^*)$, this contradicts Assumption~\ref{A1}.

Therefore, if $|A| = \infty$, then for each $i$, there exists $n_i\in \mathbb{Z}$, such that for any $k\geq n_i$,
$$g_i(y^k) + \langle \nabla g_i(y^k),x^{k+1} - y^k \rangle\leq 0.$$
Letting $N_0 =  \max_{i = 1, \dots, m} n_i$. Then for all $i\in\{1, 2, \dots, m\}$ and for any $k\ge N_0$, we have
$$g_i(y^k) + \langle \nabla g_i(y^k),x^{k+1} - y^k \rangle\leq 0.$$
From the definition of $\theta_k$ in Step 3 of Algorithm \ref{alg:Framwork}, we have $\theta_k\equiv \theta_{N_0}$ for all $k\geq N_0$, which contradicts $\theta_k\rightarrow\infty$. Thus, we have $|A|\not= \infty$.

Since $|A|$ is finite, there exists $N_0\in \mathbb{Z}$, such that $\theta_k\equiv\theta_{N_0}$ whenever $k\geq N_0$. From Step 3 of Algorithm \ref{alg:Framwork}, we known that for each $i$, $ g_i(y^k) + \langle \nabla g_i(y^k), x^{k+1} - y^k \rangle\leq 0$, for all $k\geq N_0$. Then by the definition of $s^{k+1}$, we see that $s^{k+1}=0$, for any $k\geq N_0$. This completes the proof.
\end{proof}

Finally, we prove that any cluster point of sequence $\{x^k\}$ is a critical point of problem \eqref{eq1}.

\begin{theorem}\label{subconver}
Consider \eqref{eq1} and suppose that Assumption \ref{A1} holds and $\bar{\beta}: = \sup_k\beta_k < \sqrt{\frac{L_g}{L_g + \ell_g}}$. Let $\{(x^k,y^k,\theta_k)\}$ be the sequence generated by Algorithm \ref{alg:Framwork}. Then the following statements hold:
\begin{enumerate}[{\rm (i)}]
    \item For any accumulation point $\bar{x}$ of $\{x^k\}$, there exists $\bar{\lambda}_i\geq 0$ for each $i\in \tilde{I}(\bar{x})$ such that $\sum_{i\in \tilde{I}(\bar{x})} \bar{\lambda}_i = 1$ and
        \begin{equation}\label{critical3333}
        0\in \partial P_1(\bar{x}) - \bar{\xi} + \theta_{N_0}\sum_{i\in \tilde{I}(\bar{x})} \bar{\lambda}_i\nabla g_i(\bar{x}) + \mathcal{N}_C(\bar{x}),
        \end{equation}
        where $\tilde{I}(\bar{x}):=\left\{s \in\{0,1,\cdots,m\}: g_s(\bar{x}) = \max_{i=0,1,\cdots,m} \{g_i(\cdot)\}\right\}$.\footnote{Notice that $\tilde{I}(\cdot)\subseteq\{0,1,2,\cdots,m\}$ and $I(\cdot)\subseteq\{1,2,\cdots,m\}$}
    \item Any accumulation point of sequence $\{x^k\}$ is a critical point of \eqref{eq1}
\end{enumerate}
\end{theorem}
\begin{proof}
(i): Suppose $\bar{x}$ is an accumulation point of $\{x^k\}$ with $\lim_{j\to \infty} x^{k_j} = \bar{x}$ for some convergence subsequence $\{x^{k_j}\}$. Since $\{\lambda^k\}$ is bounded thanks to Lemma~\ref{subproremarks}(iii) and $\{\xi^k\}$ is bounded thanks to $P_2$ is convex and \cite[Theorem~2.6]{Tu98}, passing to a further subsequence if necessary, we may assume without loss of generality that $\lim_{j\to \infty} \lambda^{k_j} = \bar{\lambda}\geq 0$ and $\lim_{j\to \infty} \xi^{k_j} = \bar{\xi}$.

Moreover, by the infiniteness of $\{(x^{k_j+1}, y^{k_j})\}$ and the finiteness of $\{I_{k_j}(x^{k_j+1})\}$ (since $I_{k_j}(x^{k_j+1})\subseteq \{0, 1, 2, \dots, m\}$ for all $j$), passing to a further subsequence if necessary, there exists a nonempty subset $I_0\subseteq \{0, 1,\dots,m\}$ such that
$I_0 \equiv I_{k_j}(x^{k_j+1})$ for all $k_j$ and $I_0\subseteq \tilde{I}(\bar{x}):=\{s = 0,1,\cdots,m: g_s(\bar{x}) = \max_{i=0,1,\cdots,m} \{g_i(\bar{x})\}\}$.

By subproblem \eqref{eq2},  we obtain that for each $k_j$,
\[
g_i(y^{k_j}) + \langle\nabla g_i(y^{k_j}), x^{{k_j}+1} - y^{k_j} \rangle \leq s^{{k_j}+1} ~~ \forall i =1,\dots,m,
\]
and by Lemma~\ref{subproremarks}(iii), there exist $\lambda_i^{k_j}\geq 0$ for each $i\in I_0$ such that $\sum_{i\in I_0} \lambda_i^{k_j} = 1$ and
\[
0\in \partial P_1(x^{k_j+1}) - \xi^{k_j} + \theta_{k_j}L_g(x^{{k_j}+1} - y^{k_j}) + \theta_{k_j} \sum_{i\in I_0} \lambda_i^{k_j}\nabla g_i(y^{k_j}) + \mathcal{N}_C(x^{{k_j}+1}).
\]
Passing to the limit in the above inequalities, note that $\lim_{j\to \infty} \| x^{k_j} - x^{k_j-1} \|=0$, $\lim_{m\to \infty} \| x^{k_j+1} - y^{k_j} \|=0$ (thanks to Theorem~\ref{suffdec}(iii)), $s^{k_j+1} = 0$ and $\theta_{k_j} \equiv \theta_{N_0}$, for any $k_j\geq N_0$ (thanks to Theorem~\ref{alpha}), we see that
\begin{equation}\label{critical1}
g_i(\bar{x})\le 0 ~~ \forall i=1,\dots,m,
\end{equation}
and by the closedness of $\partial P_1$, $\partial P_2$ and $\mathcal{N}_C$ (thanks to $P_1,P_2:\R^n\to\R$ are convex and \cite[Theorem~24.7]{Ro70}), there exist $\bar{\lambda}_i\geq 0$ for all $i\in I_0$ such that $\sum_{i\in I_0} \bar{\lambda}_i = 1$ and
\begin{equation}\label{critical3}
0\in \partial P_1(\bar{x}) - \bar{\xi} + \theta_{N_0}\sum_{i\in I_0} \bar{\lambda}_i\nabla g_i(\bar{x}) + \mathcal{N}_C(\bar{x}).
\end{equation}
For all $i\in \tilde{I}(\bar{x})\setminus I_0$, let $\bar{\lambda}_i = 0$, we see that (i) holds.

(ii): Letting $\hat{\lambda}_i = \theta_{N_0}\bar{\lambda}_i\geq 0$, for all $i\in I_0\cap\{1,2,\cdots,m\}$, and $\hat{\lambda}_i = 0$, for all $i\in \{1,2,\cdots,m\}\setminus I_0$. Then by \eqref{critical1} and $I_0\subseteq \tilde{I}(\bar{x})$, we have that
\begin{equation}\label{critical2}
\hat{\lambda}_i g_i(\bar{x}) = 0 ~~ \forall i=1,\dots,m.
\end{equation}
Indeed, for each $i\in I_0$, we have that $g_i(\bar{x}) = 0$, and for each $i\notin I_0$, we have that $\hat{\lambda}_i = 0$.

Notice that $\nabla g_0 (\bar{x}) = 0$ (thanks to $g_0 \equiv 0$), by the definition of $\hat{\lambda}_i$ and \eqref{critical3}, we have that
\begin{equation}\label{critical33}
\begin{aligned}
0&\in \partial P_1(\bar{x}) - \bar{\xi} + \sum_{i=1}^m \hat{\lambda}_i\nabla g_i(\bar{x}) + \mathcal{N}_C(\bar{x})\\
&\subseteq\partial P_1(\bar{x}) - \partial P_2(\bar{x}) + \sum_{i=1}^m \hat{\lambda}_i\nabla g_i(\bar{x}) + \mathcal{N}_C(\bar{x}).
\end{aligned}
\end{equation}
Combining \eqref{critical1}, \eqref{critical2} and \eqref{critical33}, we see that $\bar{x}$ is a critical point of \eqref{eq1}. This completes the proof.
\end{proof}


%\subsection{Global convergence}

We next consider the global convergence property of the sequence $\{x^k\}$ generated by Algorithm \ref{alg:Framwork}. For convenience, we let
\begin{equation}\label{defH}
H(x,y,z) := \frac{P(x) - \bar{m}}{\hat{\theta}} + \delta_{C}(x) + \max_i[\ling{i}{x}{z}]_+ + \frac{L_g}{2}\| x-y \|^2 + \frac{L_g}{2}\| x - z \|^2,
\end{equation}
where $\hat{\theta}:= \theta_{N_0}$, which was defined in Theorem \ref{alpha} and $\bar{m}$ is defined in Theorem \ref{suffdec}(ii).

\begin{remark}\label{rebarh}
In fact, from the definition of $Q$ in Theorem \ref{suffdec}~{\rm (ii)}, we see that $H(x,y,z) = Q(x,y,z,\theta_{N_0})$. According to Theorem \ref{alpha}, there exists $N_0\in \mathbb{R}$ such that $\theta_k \equiv \theta_{N_0}$ and $s^{k+1} = 0$ for all $k\geq N_0$. Thus, we have $H(x^{k+1},x^{k},y^k) = Q(x^{k+1},x^{k},y^k,\hat{\theta})$ for all $k\geq N_0$. Then one can see that the sequence $\{H(x^{k+1},x^{k},y^k)\}$ is nonincreasing thanks to Theorem \ref{suffdec}~{\rm (ii)}. Moreover, for all $k\geq N_0$,
$$
H(x^{k+1},x^{k},y^{k}) \leq H(x^{k},x^{k-1},y^{k-1}) - \frac{L - (L + \ell)\bar{\beta}^2}{2}\| x^{k} - x^{k-1}\|^2,
$$
where $\bar{\beta}: = \sup_k\beta_k$.
\end{remark}

\begin{lemma}\label{lemma1}
Consider \eqref{eq1} and suppose that Assumption~\ref{A1} holds, and $\bar{\beta}: = \sup_k\beta_k < \sqrt{\frac{L_g}{L_g + \ell_g}}$. Let $\{(x^k,y^k)\}$ be the sequence generated by Algorithm \ref{alg:Framwork}, and $\Omega$ be the set of accumulation points of $\{(x^{k+1},x^{k},y^{k})\}$. Then $\Omega$ is a nonempty compact set, $\omega := \lim_{k\to \infty} H(x^{k+1},x^{k}, y^{k})$ exists, and $H \equiv \omega$ on $\Omega$.
\end{lemma}
\begin{proof}
From Theorem \ref{suffdec}(i), we have that the set of accumulation points of $\{x^k\}$, denoted by $\varLambda$, is a nonempty compact set. Since $\lim_{k\to \infty} \| x^k - x^{k-1} \|=0$ and $\lim_{k\to \infty} \| x^k - y^k\| = 0$ (see Theorem \ref{suffdec}(iii)), one can see that $\Omega= \{(\bar{x},\bar{x},\bar{x}): \bar{x}\in \varLambda\}$ is a nonempty compact set.

Thanks to Remark~\ref{rebarh}, we obtain that for any $k\geq N_0$, $\{H(x^{k+1},x^{k}, y^{k})\}$ is nonincreasing. According to the definition of $H$ (see \eqref{defH}), we obtain that for any $x\in C$, $y, z\in \mathbb{R}^n$, $H(x,y,z)\geq 0$, which implies that $\{H(x^{k+1},x^{k}, y^{k})\}$ is bounded from below. We deduce that $\omega := \lim_{k\to \infty} H(x^{k+1},x^{k}, y^{k})$ exists.

For any $(\bar{x},\bar{x},\bar{x})\in \Omega$, let $\{x^{k_j}\}$ be a convergent subsequence with $\lim_{j\to \infty} x^{k_j} = \bar{x}$.
Since $P$ and $g_i$ are continuous, $\lim_{k\to \infty} \| x^k - x^{k-1}\| = 0$ and $\lim_{k\to \infty} \| x^k - y^{k}\| = 0$ (see Theorem~\ref{suffdec}(iii)), we obtain that
\begin{align*}
& H(\bar{x}, \bar{x}, \bar{x}) = \frac{P(\bar{x}) - \bar{m}}{\hat{\theta}} + \max_{i = 1,\cdots,m}\left[\ling{i}{\bar x}{\bar x}\right]_+\\
&\!\!=\!\lim_{j\to \infty}\! \frac{P(x^{k_j+1}) \!-\! \bar{m}}{\hat{\theta}} \!+\!\!\max_{i = 1,\cdots,m}\!\left[\ling{i }{x^{k_j+1}}{y^{k_j}}\right]_+ \!\!\!+\! \frac{L_g}{2}\| x^{k_j+1} \!\!-\! x^{k_j} \|^2 \!+\! \frac{L_g}{2}\| x^{k_j+1} \!\!-\! y^{k_j} \|^2 \\
&\!\!=\lim_{j\to \infty} H(x^{k_j+1},x^{k_j},y^{k_j}).
\end{align*}

Then for any $\bar x\in \Omega$, we have
\[
H(\bar{x}, \bar{x}, \bar{x}) = \lim_{j\to \infty} H(x^{k_j+1},x^{k_j},y^{k_j}) = \omega.
\]
Since $(\bar{x}, \bar{x}, \bar{x})\in \Omega$ is arbitrary, we conclude that $H \equiv \omega$ on $\Omega$.
%This completes the proof.
\end{proof}


Now, we introduce the following assumption to derive a bound on $\partial H (x^{k+1}, x^k, y^k)$. This assumption appears in \cite{wen18,yu21} and is satisfied in many applications; see \cite{wen18}.

\begin{assumption}\label{A2}
Each $g_i$ in \eqref{eq1} is twice continuously differentiable. The function $P_2$ be continuously differentiable on an open set $U_0$ which contains $\mathcal{X}$, and the gradient $\nabla P_2$ be locally Lipschitz continuous on $U_0$, where $\mathcal{X}$ be the set of critical points of \eqref{eq1}.
\end{assumption}

Now, we present the following property of $\partial H$.

\begin{lemma}\label{th2.1}
Consider \eqref{eq1} and suppose that Assumptions~\ref{A1} and \ref{A2} hold, and $\bar{\beta}: = \sup_k\beta_k < \sqrt{\frac{L_g}{L_g + \ell_g}}$. Let $\{(x^k,y^k)\}$ be a sequence generated by Algorithm \ref{alg:Framwork}. Then, we have that
\[
\lim_{k\to \infty}\d ((0,0,0),\partial H(x^{k+1},x^{k},y^{k}))=0.
\]
%\begin{enumerate}[{\rm (i)}]
%    \item
%    \item The sequence $\{x^k\}$ globally converges to a critical point of \eqref{eq1}. Moreover, $\sum_{k=1}^{\infty} \| x^{k+1} - x^{k}\|<\infty.$
%\end{enumerate}
\end{lemma}
\begin{proof}
Let $\varLambda$ be the set of accumulation points of $\{x^k\}$. By the boundedness of $\{x^k\}$ (see Theorem~\ref{suffdec}(i)), we have that $\varLambda$ is nonempty and compact, and $\varLambda\subseteq\mathcal{X}$ (see Theorem~\ref{subconver}(ii)). Then, we have $\lim_{k\to \infty}\text{dist}(x^k,\varLambda)=0$. Thus, for any $\gamma>0$, there exists $N_1>0$, such that dist$(x^k,\varLambda)<\gamma$ and $x^k\in U_0$ (where $U_0$ was defined as in Assumption \ref{A2}) for all $k\geq N_1$.
%Since $\varLambda$ is a nonempty compact set, by shrinking $\gamma$ if necessary, we denote the bounded set $U := \{x\in U_0:\text{dist}(x,\varLambda)<\gamma\}$. WLOG, we assume that $\nabla P_2$ is globally Lipschitz continuous on $U$.

Now, considering the subdifferential of $H$ at the point $(x^{k+1},x^{k},y^{k})$, for any $k\geq \max\{N_0,N_1\}$, where $N_0$ was defined in Theorem \ref{alpha}. Since $P_2$ is continuously differentiable on $U_0$ and $x^k\in U_0$ for any $k\geq \max\{N_0,N_1\}$, we obtain from \cite[Theorem~8.6]{rock97a} that
\begin{align}\label{eq16}
&\!\!\!\!\!\!\!\!\!\!~~~\partial H(x^{k+1},x^{k},y^{k})\supseteq \widehat{\partial} H(x^{k+1},x^{k},y^{k})\notag \\
&\!\!\!\!\!\!\!\!\!\!~~\overset{\rm(a)}\supseteq\!\!\!
\left[\!
\begin{array}{c}
\Xi_k    \\ [2 pt]
 - L_g(x^{k+1} - x^{k}) \\ [2 pt]
 \conv_{i\in I_k(x^{k+1})}\{\nabla^2 g_i(y^{k})(x^{k+1} - y^{k})\} - L_g(x^{k+1}-y^{k})
\end{array}\!
\right] \notag\\
&\!\!\!\!\!\!\!\!\!\!~~\overset{\rm(b)}{\supseteq}\!\!\!
\left[\!
\begin{array}{c}
\!\frac{1}{\hat{\theta}}\partial P(x^{k+1}) \!+ \!\!\!\!\sum\limits_{i\in I_k(x^{k+1})}\!\!\! \lambda_i^k \nabla g_i(y^k) \!+\! \mathcal{N}_C(x^{k+1}) \!+\! L_g(x^{k+1} \!-\! x^k) \!+\! L_g(x^{k+1} \!-\! y^k)    \\ 
 - L_g(x^{k+1} - x^{k}) \\ [5 pt]
\sum_{i\in I_k(x^{k+1})} \lambda_i^k\nabla^2 g_i(y^{k})(x^{k+1} - y^{k}) - L_g(x^{k+1}-y^{k})
\end{array}\!\!
\right]\!\!,
\end{align}
where
\[
\Xi_k := \frac{1}{\hat{\theta}}\partial P(x^{k+1}) + \conv_{i\in I_k(x^{k+1})}\{\nabla g_i(y^k)\} + \mathcal{N}_C(x^{k+1}) + L_g(x^{k+1} - x^k) + L_g(x^{k+1} - y^k),
\]
and (a) holds because of the subdifferential calculus rules in \cite[Proposition~10.5, Corollary~10.9, Exercise~8.31]{rock97a} and the regularity of the closed convex convex set $C$ and the convex function $P_1$ (thanks to \cite[Corollary~10.9, Proposition~8.12, Exercise~8.8]{rock97a}), and (b) holds as $\sum_{i\in I_k(x^{k+1})} \lambda_i^k = 1$, where $I_k(x^{k+1})$ and $\lambda_i^k$ were defined as in Lemma~\ref{subproremarks}(iii).
%$(c)$ holds for any $\lambda^k \in \widehat{\mathcal{N}}_{-\mathbb{R}_+^m}(\bar{g}(x^{k+1}, y^k)) = \mathcal{N}_{-\mathbb{R}_+^m}(\bar{g}(x^{k+1}, y^k))$ thanks to \cite[Theorem 6.14]{rock97a}.

On the other hand, by Lemma~\ref{subproremarks}(iii) and Theorem \ref{alpha}, we have that, for any $k\ge N_0$,
\begin{align*}
0\in \partial P_1(x^{k+1}) - \nabla P_2(x^{k}) + \hat{\theta}\sum_{i\in I_k(x^{k+1})} \lambda_i^k \nabla g_i(y^k) + \hat{\theta}L_g(x^{k+1} - y^{k})+ \mathcal{N}_C(x^{k+1}).
\end{align*}
Rearranging terms in the above display, we see that
\begin{align}\label{eq23}
\nabla P_2(x^{k}) \!-\! \hat{\theta}\sum_{i\in I_k(x^{k+1})} \lambda_i^k \nabla g_i(y^k) - \hat{\theta}L_g(x^{k+1} - y^{k})\in \partial P_1(x^{k+1}) + \mathcal{N}_C(x^{k+1}).
\end{align}

We claim that for any $k\geq\max\{N_0, N_1\}$,
\begin{align}\label{eq22}
&\frac{1}{\hat{\theta}}\left(-\hat{\theta}L_g(x^{k} - y^{k}) + \nabla P_2(x^{k}) - \nabla P_2(x^{k+1})\right)  \nonumber\\
& \in \frac{1}{\hat{\theta}}\partial P(x^{k+1}) + \sum_{i\in I_k(x^{k+1})} \lambda_i^k \nabla g_i(y^k) + \mathcal{N}_C(x^{k+1}) + L_g(x^{k+1} - x^k).
\end{align}
In fact, since $P_2$ is continuously differentiable in $U_0$, we obtain that
\begin{align*}
&\frac{1}{\hat{\theta}}\left(-\hat{\theta}L_g(x^{k} - y^{k}) + \nabla P_2(x^{k}) - \nabla P_2(x^{k+1})\right) \notag\\
& = \frac{1}{\hat{\theta}}\left(\hat{\theta}L_g(x^{k+1} - x^{k}) - \nabla P_2(x^{k+1}) + \hat{\theta}\sum_{i\in I_k(x^{k+1})} \lambda_i^k\nabla g_i(y^k)\right) \notag\\
&~~~~ + \frac{1}{\hat{\theta}}\left(\nabla P_2(x^{k}) - \hat{\theta}\sum_{i\in I_k(x^{k+1})} \lambda_i^k\nabla g_i(y^k)
 - \hat{\theta}L_g(x^{k+1} - y^{k})\right) \\
& \overset{\rm(a)}{\in}\! \frac{1}{\hat{\theta}}\!\left(\hat{\theta}L_g(x^{k+1} \!-\! x^{k}) \!-\! \nabla P_2(x^{k+1}) + \hat{\theta}\!\!\!\!\sum_{i\in I_k(x^{k+1})}\!\! \lambda_i^k\nabla g_i(y^k)\right) \!+\! \frac{1}{\hat{\theta}}\partial P_1(x^{k+1}) \!+\! \mathcal{N}_C(x^{k+1}) \notag\\
& =\frac{1}{\hat{\theta}}\left(\partial P_1(x^{k+1}) - \nabla P_2(x^{k+1})\right) + \sum_{i\in I_k(x^{k+1})} \lambda_i^k\nabla g_i(y^k) + \mathcal{N}_C(x^{k+1}) + L_g(x^{k+1} - x^{k}) \notag,
\end{align*}
where (a) holds because of \eqref{eq23}.

Combining \eqref{eq16} and \eqref{eq22}, for any $k\geq \max\{N_0, N_1\}$, we have that
\begin{equation}
\begin{aligned}
\nonumber
\left[
\begin{array}{c}
\frac{1}{\hat{\theta}}\left(-\hat{\theta}L_g(x^{k} - y^{k}) + \nabla P_2(x^{k}) - \nabla P_2(x^{k+1})\right) + L_g(x^{k+1} - y^{k})\\
-L_g(x^{k+1} - x^{k})\\
\sum_{i\in I_k(x^{k+1})}\lambda_i^k\nabla^2 g_i(y^{k})(x^{k+1} - y^{k}) - L_g(x^{k+1} - y^{k})
\end{array}
\right]
\in \partial H(x^{k+1},x^{k},y^{k}).
\end{aligned}
\end{equation}
Since $\nabla P_2$ is Lipschitz continuous with modulus $L_{P_2}$, $y^k = x^k + \beta_k(x^k - x^{k-1})$ (see \eqref{defyk}), we see that for any $k\geq \max\{N_0,N_1\}$,
\begin{align*}
& \d^2\left((0,0,0),\partial H(x^{k+1},x^{k},y^{k})\right) \\
&\leq \left\| \frac{1}{\hat{\theta}}\left(\!-\hat{\theta}L_g(x^{k} - y^{k}) + \nabla P_2(x^{k}) - \nabla P_2(x^{k+1})\right) + L_g(x^{k+1} \!-\! y^{k})\right\|^2\!\!\!\! +\! \| L_g(x^{k+1} - x^{k})\|^2 \\
&~~~~~~ + \left\| \sum_{i\in I_k(x^{k+1})} \lambda_i^k\nabla^2 g_i(y^k)(x^{k+1} - y^{k}) - L_g(x^{k+1} - y^{k})\right\|^2 \\
&\overset{\rm(a)}{\leq} 3L_g^2\| x^{k} - y^{k}\|^2 + \frac{3}{\hat{\theta}^2}L_{P_2}^2\| x^{k+1} - x^{k}\|^2 + 3L_g^2\| x^{k+1} - y^{k}\|^2 + L_g^2\| x^{k+1} - x^{k} \|^2 \\
&~~~~~~ + 2\left(\sum_{i\in I_k(x^{k+1})} \lambda_i^k\nabla^2 g_i(y^k)\right)^2 \| x^{k+1}-y^{k} \|^2 + 2L_g^2\| x^{k+1} - y^{k}\|^2,
\end{align*}
where (a) holds for the Cauchy-Schwarz inequality.  Hence, by the boundedness of $\{y^k\}$ (thanks to Theorem~\ref{suffdec} and \eqref{defyk}) and the continuity of $\nabla^2 g_i$ ( thanks to Assumption~\ref{A2}), we have that for any $k\geq \max\{N_0,N_1\}$, there exists $T > 0$, such that
\begin{equation} \label{dist}
\d^2\left((0,0,0),\partial H(x^{k+1},x^{k},y^{k})\right) \leq T\left(\|x^{k+1} - x^{k}\|^2 + \|x^{k} - x^{k-1}\|^2\right).
\end{equation}
Since $\lim_{k\to \infty}\| x^k - x^{k-1} \| = 0$ (thanks to Theorem \ref{suffdec}~(iii)), we see that
$$ \lim_{k\to \infty}\d \left((0,0,0),\partial H(x^{k+1},x^{k},y^{k})\right)=0.$$
This completes the proof.
\end{proof}

Now, we present the convergence rate of the sequence $\{x^k\}$ under the assumption that the function $H$ is a KL function. The proof of Theorem \ref{th2.2} is routine and we refer to \cite[Theorem 4.3]{wen18}.
\begin{theorem}\label{th2.2}
Consider \eqref{eq1} and suppose that Assumptions~\ref{A1} and \ref{A2} hold, $\bar{\beta}: = \sup_k\beta_k < \sqrt{\frac{L_g}{L_g + \ell_g}}$, and $H$ in \eqref{defH} is a KL function. Let $\{(x^k, y^k)\}$ be a sequence generated by Algorithm \ref{alg:Framwork} and let $\Omega$ be the set of accumulation points of $\left\{(x^{k+1},x^{k},y^{k})\right\}$. Then $\{x^k\}$ converges to a critical point $\bar{x}$ of \eqref{eq1}. Moreover, if $H$ satisfies the KL property with exponent $\alpha\in [0,1)$ at every point in $\Omega$, then there exists $\underline{N}\in\mathbb{N}$ such that the following statements hold:
\begin{enumerate}[{\rm (i)}]
    \item If $\alpha=0$, then $\{x^k\}$ converges finitely.
    \item If $\alpha\in (0,\frac{1}{2}]$, then there exist $a_0\in(0, 1)$ and $a_1>0$ such that \[
        \|x^k - \bar{x}\|\leq a_1a_0^k ~~\forall k > \underline{N}.
        \]
    \item If $\alpha\in (\frac{1}{2},1)$, then there exists $a_2>0$ such that
        \[
        \|x^k - \bar{x}\|\leq a_2k^{-\frac{1-\alpha}{2\alpha-1}} ~~\forall k > \underline{N}.
        \]
\end{enumerate}
\end{theorem}


\subsection{Convergence analysis in convex setting}
In this section, we study the convergence properties of Algorithm \ref{alg:Framwork} under the following convex settings.

\begin{assumption}\label{B1}
Suppose that in \eqref{eq1}, $P_2 = 0$ and $\{g_1, . . . , g_m\}$ are convex.
\end{assumption}

\begin{assumption}\label{B2}
The Slater condition holds for $\mathcal{G}$ in \eqref{eq1}, i.e., there exists $\hat{x}\in C$ with $g_i(\hat{x})<0$ for $i=1,\dots,m$.
\end{assumption}
\begin{remark}
If each $g_i$ is convex and Assumption~\ref{B2} holds, then $RCQ(x)$ holds at every point $x\in C$, which implies that Assumption~\ref{A1} holds thanks to Remark~\ref{remarkRCQ}.
\end{remark}

Now, we present the convergence properties of Algorithm \ref{alg:Framwork} under the Assumptions~\ref{B1} and \ref{B2}.
\begin{theorem}[{{\bf Convergence rate of Algorithm \ref{alg:Framwork} in convex setting}}]\label{Edecres}
Consider \eqref{eq1} and suppose that Assumptions~\ref{B1} and \ref{B2} hold, and $\bar{\beta}: = \sup_k\beta_k < 1$. Let $\{(x^k,\theta_k)\}$ be the sequence generated by Algorithm \ref{alg:Framwork}. Then following statements hold:
\begin{enumerate}[{\rm (i)}]
    \item Let $\hat{m} = \min\{P_1(x): x\in C\}$. Then, for any $k>0$,
          $$E(x^{k+1},x^{k},\theta_{k+1}) \leq E(x^k,x^{k-1},\theta_k) - \frac{(1 - \beta_k^2)L_g}{2}\|x^{k} - x^{k-1}\|^2,$$
          where $E(x,y,\theta):=\frac{1}{\theta}\left(P_1(x) - \hat{m} + \theta\max_{i = 1,\cdots,m}\left[g_i(x)\right]_+ + \frac{\theta L_g}{2}\| x - y \|^2\right)$ and $L_g$ was defined in Remark~\ref{Remarkg}.
    \item Let $\Omega$ be the set of accumulation points of $\left\{(x^{k+1},x^{k},\theta_k)\right\}$. Then $\Omega$ is a nonempty compact set, $\bar{\omega} := \lim_{k\to \infty} E(x^{k+1},x^{k}, \theta_k)$ exists, and $E \equiv \bar{\omega}$ on $\Omega$.
    \item If $x\mapsto \frac{1}{\hat{\theta}}\left(P_1(x) + \hat{\theta}\max_{i = 1,\cdots,m}\left[g_i(x)\right]_+ + \delta_C(x)\right)$ is a KL function with exponent $\frac{1}{2}$, where $\hat{\theta}:= \theta_{N_0}$ defined in Theorem~\ref{alpha}, then $\{x^k\}$ converges to a minimizer $x^*$ of \eqref{eq1}, and there exist $c_0 > 0$, $s\in (0,1)$ and $k_0\in \mathbb{N_+}$, such that
        \[
        \|x^k - x^*\|\leq c_0 s^k ~~\forall k > k_0.
        \]
\end{enumerate}
\end{theorem}
%
%\begin{proof}
%(i): Since $\{x^k\}\subseteq C$ and $C$ is a nonempty compact convex set, $\{x^k\}$ is bounded.
\begin{proof}
Note that $s^{k+1} = \max_{i = 1,\cdots,m}\left[\ling{i}{x^{k+1}}{y^k}\right]_+$ (thanks to Lemma~\ref{subproremarks}(i)), using the definition of $(x^{k+1},s^{k+1})$ in \eqref{eq2} and the strong convexity of the objective in the minimization problem \eqref{eq2}, we obtain that for any $x\in C$,
\begin{align}\label{stongconve}
&P_1(x^{k+1}) + \theta_k \max_{i = 1,\cdots,m}\left[\ling{i}{x^{k+1}}{y^k}\right]_+ + \frac{\theta_k L_g}{2}\| x^{k+1} - y^k\|^2 \notag\\
&= P_1(x^{k+1}) + \theta_k s^{k+1} + \frac{\theta_k L_g}{2}\| x^{k+1} - y^k\|^2 \\
&\leq P_1(x) + \theta_k \max_{i = 1,\cdots,m}\left[\ling{i}{x}{y^k}\right]_+ + \frac{\theta_k L_g}{2}\| x - y^k \|^2 - \frac{\theta_k L_g}{2}\| x - x^{k+1}\|^2. \notag
\end{align}
Now, we prove item (i).

(i): For any $k > 0$, we see that
\begin{align*}
&E(x^{k+1},x^{k},\theta_{k+1}) = \frac{1}{\theta_{k+1}}\left(P_1(x^{k+1}) - \hat{m}\right) + \max_{i = 1,\cdots,m}\left[g_i(x^{k+1})\right]_+ + \frac{L_g}{2}\| x^{k+1} - x^k \|^2\\
&\overset{\rm(a)}{\leq} \frac{1}{\theta_k}\left(P_1(x^{k+1}) - \hat{m}\right) + \max_{i = 1,\cdots,m}\left[g_i(x^{k+1})\right]_+ + \frac{L_g}{2}\| x^{k+1} - x^k \|^2\\
&\overset{\rm(b)}{\leq} \frac{P_1(x^{k+1}) - \hat{m}}{\theta_k} + \max_{i = 1,\cdots,m}\left[\ling{i}{x^{k+1}}{y^k} + \frac{L_{g_i}}{2}\| x^{k+1} - y^k \|^2\right]_+\!\!\!\!+\!\! \frac{L_g}{2}\| x^{k+1} - x^k \|^2\\
&\overset{\rm(c)}{\leq} \frac{P_1(x^{k+1}) - \hat{m}}{\theta_k} + \max_{i = 1,\cdots,m}\left[\ling{i}{x^{k+1}}{y^k} \right]_+ + \frac{L_{g}}{2}\| x^{k+1} - y^k \|^2 + \frac{L_g}{2}\| x^{k+1} - x^k \|^2\\
&\overset{\rm(d)}{\leq}\left(\frac{P_1(x^{k}) - \hat{m}}{\theta_k} + \max_{i = 1,\cdots,m}\left[\ling{i}{x^k}{y^k}\right]_+ + \frac{L_{g}}{2}\| x^{k} - y^k \|^2 - \frac{L_g}{2}\| x^{k+1} - x^k \|^2 \right) \\
& ~~~~ + \frac{L_g}{2}\| x^{k+1} - x^k \|^2\\
&\overset{\rm(e)}{\leq}\left(\frac{P_1(x^{k}) - \hat{m}}{\theta_k} +  \max_{i = 1,\cdots,m}\left[g_i(x^k)\right]_+ + \frac{L_{g}}{2}\| x^{k} - y^k \|^2 - \frac{L_g}{2}\| x^{k+1} - x^k \|^2 \right)\\
&~~~~ + \frac{L_g}{2}\| x^{k+1} - x^k \|^2\\
&= \frac{1}{\theta_k}\left(P_1(x^{k}) - \hat{m} + \theta_k\max_{i = 1,\cdots,m}\left[g_i(x^k)\right]_+\right) + \frac{\beta_k^2 L_g}{2}\|x^k - x^{k-1}\|^2\\
&= E(x^{k},x^{k-1},\theta_k) - \frac{(1 - \beta_k^2)L_g}{2}\|x^k - x^{k-1}\|^2,
\end{align*}
where (a) holds thanks to $\theta_k \leq\theta_{k+1}$ and $\hat{m} = \min\{P_1(x): x\in C\}$, (b) holds because of the Lipschitz continuity of $\nabla g_i$, (c) follows from $L_g := \max\{L_{g_i}:\; i=1,\dots,m\}$, (d) holds thanks to \eqref{stongconve} with $x=x^k$ (as $x^k\in C$), and (e) follows from the convexity of $g_i$.

(ii): Using the similar arguments as Lemma~\ref{lemma1}, one can see that (ii) holds. We omit its proof for brevity.

(iii): For notational simplicity, we defined
\begin{equation}\label{definfalpha}
F_{\hat{\theta}}(x):= \frac{1}{\hat{\theta}}\left(P_1(x) - \hat{m} + \hat{\theta}\max_{i = 1,\cdots,m}\left[g_i(x)\right]_+ + \delta_C(x)\right),
\end{equation}
where $\hat{\theta}: = \theta_{N_0}$, which was defined in Theorem~\ref{alpha}.

Since $C$ is a compact set and each $g_i$ is continuous, the feasible set ${\cal G}$ of \eqref{eq1} is compact. Combining this with the fact that $P_1$ is continuous, we have $S:= \Argmin_{x\in \mathcal{G}} P_1(x) = \Argmin F_{\hat{\theta}}(x) \not=\emptyset$. Let $\Lambda$ be the set of accumulation point of $\{x^k\}$ for notational simplicity. Using Theorem \ref{subconver}(ii) and \cite[Theorem~28.3]{Ro70}, we see that
\begin{equation}\label{set}
\emptyset\not=\Lambda\subseteq S.
\end{equation}

Let $\bar{x}^k\in S$ satisfy $\|x^k - \bar{x}^k\| = \d(x^k, S)$. Then for any $k\geq N_0$ (which was defined in Theorem~\ref{alpha}) and $\gamma\in(0, 1)$, we have
\begin{align}\label{upperF}
&F_{\hat{\theta}}(x^{k+1})= \frac{1}{\hat{\theta}}\left(P_1(x^{k+1}) - \hat{m} + \hat{\theta}\max_{i = 1,\cdots,m}\left[g_i(x^{k+1})\right]_+\right) \nonumber\\
&\overset{\rm (a)}\leq \frac{1}{\hat{\theta}}\left(P_1(x^{k+1}) - \hat{m} + \hat{\theta}\max_{i = 1,\cdots,m}\left[\ling{i}{x^{k+1}}{y^k} + \frac{L_{g_i}}{2}\|x^{k+1} - y^k\|^2\right]_+\right) \nonumber\\
&\overset{\rm (b)}\leq\frac{1}{\hat{\theta}}\left(P_1(x^{k+1}) - \hat{m} + \hat{\theta}\max_{i = 1,\cdots,m}\left[\ling{i}{x^{k+1}}{y^k}\right]_+ + \frac{\hat{\theta}L_{g}}{2}\|x^{k+1} - y^k\|^2\right)\nonumber\\
&\overset{\rm (c)}\leq\frac{1}{\hat{\theta}}\left(P_1(\bar{x}^k) - \hat{m}\right) + \frac{L_g}{2}\|\bar{x}^k - y^k\|^2 - \frac{L_g}{2}\|\bar{x}^k - x^{k+1}\|^2 \nonumber\\
&\overset{\rm (d)}\leq F_{\hat{\theta}}(\bar{x}^k) + \frac{L_g}{2}\left(\|\bar{x}^k - x^k\| + \|x^k - y^k\|\right)^2 - \frac{L_g}{2}\|\bar{x}^k - x^{k+1}\|^2, \nonumber\\
&\overset{\rm(e)}\leq F_{\hat{\theta}}(\bar{x}^k)  + \frac{L_g}{2\gamma}\|\bar{x}^k - x^k\|^2 + \frac{L_g}{2(1 - \gamma)}\|x^k - y^k\|^2 - \frac{L_g}{2}\|\bar{x}^k - x^{k+1}\|^2\\
&\overset{\rm(f)}\leq F_{\hat{\theta}}(\bar{x}^k) + \frac{L_g}{2\gamma}\d^2(x^k, S) + \frac{L_g}{2(1 - \gamma)}\|x^k - y^k\|^2 - \frac{L_g}{2}\d^2(x^{k+1}, S),\nonumber
\end{align}
where (a) holds because of the Lipschitz continuity of $\nabla g_i$, (b) holds because $L_g = \max_{i = 1,\cdots,m}\{L_{g_i}\}$, (c) follows from \eqref{stongconve} with $x = \bar{x}^k$ (thanks to $\bar{x}^k\in S\subseteq C$) and $\ling{i}{\bar{x}^k}{y^k}\leq g_i(\bar{x}^k) \leq 0$ for each $i$ (thanks to the convexity of $g_i$ and $\bar{x}^k\in S \subseteq\mathcal{F}$), (d) holds because $g_i(\bar{x}^k)\leq 0$ for each $i$ (thanks to $\bar{x}^k\in S \subseteq\mathcal{F}$), and the triangle inequality, (e) follows from the fact that $(a + b)^2 = (\gamma\frac{a}{\gamma} + (1 - \gamma)\frac{b}{(1-\gamma)})^2\leq\frac{a^2}{\gamma} + \frac{b^2}{(1-\gamma)}$, for any $\gamma\in(0,1)$, and (f) holds because $\bar{x}^k\in S$.


Write $E_{\theta}(x, y) := E(x, y, \theta)$ for notational simplicity. By the definition of $F_{\hat{\theta}}(x)$ in \eqref{definfalpha} and $E(x, y, \theta)$ in (i), we see that $E_{\hat{\theta}}(x, y) = F_{\hat{\theta}}(x) + \frac{L_g}{2}\|x - y\|^2$. From (i), we have that for any $k\geq N_0$,
\begin{equation}\label{upperK}
E_{\hat{\theta}}(x^{k+1}, x^k)\leq E_{\hat{\theta}}(x^k, x^{k-1}) -\frac{L_g(1 - \bar{\beta}^2)}{2}\|x^k - x^{k-1}\|^2.
\end{equation}
Moreover, since $x\mapsto \frac{1}{\hat{\theta}}\left(P_1(x) + \hat{\theta}\max_{i = 1,\cdots,m}\left[g_i(x)\right]_+ + \delta_C(x)\right)$ is a KL function with exponent $\frac{1}{2}$, one can see that $F_{\hat{\theta}}$ is a KL function with exponent $\alpha = \frac{1}{2}$ (see \eqref{definfalpha}). Furthermore, from \cite[Theorem~3.6]{li18}, we have that $E_{\hat{\theta}}(x, y)$ is a KL function with exponent $\frac{1}{2}$.

Let $\bar{S} = \{(x^*, x^*): x^*\in S\}$ and $\bar{\Lambda} = \{(x^*, x^*): x^*\in \Lambda\}$. Then by the definition of $S$, we have that $\bar{S} = \Argmin E_{\hat{\theta}}(x, y)$, and
\begin{equation}\label{set2}
\emptyset\not=\bar{\Lambda}\subseteq \bar{S}.
\end{equation}
Using Lemma~\ref{KLinequ}, we have that there exist $\epsilon_0 >0$, $r_0>0$, and $c_0>0$ such that
\begin{equation}\label{erro}
\d^2((x, y), \bar{S})\leq c_0(E_{\hat{\theta}}(x, y) - \bar{E}_{\hat{\theta}}^*),
\end{equation}
for any $(x, y)\in\dom \partial E_{\hat{\theta}}$ satisfying $\d((x, y), \bar{S})\leq \epsilon_0$ and $\bar{E}_{\hat{\theta}}^*\leq E_{\hat{\theta}}(x, y) \leq \bar{E}_{\hat{\theta}}^* + r_0$, where $\bar{E}_{\hat{\theta}}^* := \inf E_{\hat{\theta}}(x, y) = E_{\hat{\theta}}(\bar{x}, \bar{x}) = \bar{\omega}$, whenever $\bar{x}\in \Lambda\subseteq S$ and $\bar{\omega}$ defined in (ii).

Clearly, $\{(x^k, x^{k-1})\}\in C\times C$ and $ \dom \partial E_{\hat{\theta}} = C\times C$. From \eqref{set2} and Theorem~\ref{subconver}(ii), we have that there exists $k_1>0$ such that
\begin{equation}\label{erro1}
\d((x^k, x^{k-1}), \bar{S})\leq\d((x^k, x^{k-1}), \bar{\Lambda})\leq\epsilon_0~~ \forall k\geq k_1.
\end{equation}

Since $E_{\hat{\theta}}(x,y) = E(x,y,\hat{\theta})$, from Theorem~\ref{alpha} and (ii), we have that there exists $k_2>0$ such that
\begin{equation}\label{Ferro1}
\bar{E}_{\hat{\theta}}^*\leq E_{\hat{\theta}}(x^k, x^{k-1})\leq\bar{E}_{\hat{\theta}}^* + r_0 ~~ \forall k\geq k_2.
\end{equation}
%where $\bar{E}_{\hat{\theta}}^* := \inf K = \frac{1}{\hat{\theta}}(P_1(\bar{x}) - \hat{m})= F_{\hat{\theta}}(\bar{x}) = E_{\hat{\theta}}(\bar{x}, \bar{x}) = \bar{\omega}$, whenever $\bar{x}\in S$ and $\bar{\omega}$ defined in (ii).

Combining \eqref{erro}, \eqref{erro1} and \eqref{Ferro1}, we concluded that for any $k\geq k_3:=\max\{k_1, k_2\}$,
\begin{equation}\label{Eerro}
\d^2(x^{k}, S)\leq\d^2((x^k, x^{k-1}), \bar{S})\leq c_0(E_{\hat{\theta}}(x^k, x^{k-1}) - \bar{E}_{\hat{\theta}}^*).
\end{equation}
Then, we have that for any $k\geq k_4:= \max\{k_3, N_0\}$,
\begin{align*}%\label{upperE}
&E_{\hat{\theta}}(x^{k+1},x^k) - \bar{E}_{\hat{\theta}}^* \overset{\rm(a)}= F_{\hat{\theta}}(x^{k+1}) - \bar{F}_{\hat{\theta}}^* + \frac{L_g}{2}\|x^{k+1} - x^k\|^2 \\
&\overset{\rm(b)}\leq \frac{L_g}{2\gamma}\d^2(x^k, S) + \frac{L_g}{2(1 - \gamma)}\|x^k - y^k\|^2 - \frac{L_g}{2\gamma}\d^2(x^{k+1}, S)  \nonumber\\
&~~~~+ \frac{L_g}{2}(\frac{1}{\gamma} - 1)\d^2(x^{k+1}, S) + \frac{L_g}{2}\|x^{k+1} - x^k\|^2 \\
&\overset{\rm(c)}\leq \left(\frac{L_g}{2\gamma}\d^2(x^k, S) - \frac{L_g}{2}\|x^k - x^{k-1}\|^2\right) - \left(\frac{L_g}{2\gamma}\d^2(x^{k+1}, S) - \frac{L_g}{2}\|x^{k+1} - x^k\|^2\right) \nonumber\\
&~~~~+ \frac{L_g\bar{\beta}^2}{2(1 - \gamma)}\|x^k - x^{k-1}\|^2 + \frac{L_g}{2}\|x^k - x^{k-1}\|^2 + \frac{L_g}{2}(\frac{1}{\gamma} - 1)c_0(E_{\hat{\theta}}(x^{k+1}, x^{k}) - \bar{E}_{\hat{\theta}}^*), 
\end{align*}
where (a) follows from $E_{\hat{\theta}}(x^{k+1},x^k) = F_{\hat{\theta}}(x^{k+1}) + \frac{L_g}{2}\|x^{k+1} - x^k\|$ and $\bar{E}_{\hat{\theta}}^* = \bar{F}_{\hat{\theta}}^*$, (b) holds because of \eqref{upperF} and $\bar{F}_{\hat{\theta}}^* = F_{\hat{\theta}}(\bar{x}^k)$ (thanks to $\bar{x}^k\in S$), and (c) follows from \eqref{Eerro}, $y^k = x^k + \beta_k(x^k - x^{k-1})$ and $\bar{\beta} = \max\beta_k$.

Now, taking $\gamma\in(\frac{L_g c_0}{1 + L_gc_0}, 1)$, we have that $\frac{L_g}{2}(\frac{1}{\gamma} - 1)c_0 < \frac{1}{2}$. Letting $\vartheta: = 1 - \frac{L_g}{2}(\frac{1}{\gamma} - 1)c_0$, then we known that $\vartheta > \frac{1}{2}$. Rearranging terms in the above display inequality, we have that
\begin{align*}\label{upperE1}
&\vartheta \left(E_{\hat{\theta}}(x^{k+1},x^k) - \bar{E}_{\hat{\theta}}^*\right) \\
& \leq \frac{L_g}{2}\left(\frac{1}{\gamma}\d^2(x^k, S) - \|x^k - x^{k-1}\|^2\right) - \frac{L_g}{2}\left(\frac{1}{\gamma}\d^2(x^{k+1}, S) -\|x^{k+1} - x^k\|^2\right) \nonumber\\
&~~~~+ \frac{L_g(1 - \gamma + \bar{\beta}^2)}{2(1 - \gamma)}\|x^k - x^{k-1}\|^2 \nonumber\\
&\overset{\rm(a)}\leq \frac{L_g}{2}\left(\frac{1}{\gamma}\d^2(x^k, S) - \|x^k - x^{k-1}\|^2\right) - \frac{L_g}{2}\left(\frac{1}{\gamma}\d^2(x^{k+1}, S) -\|x^{k+1} - x^k\|^2\right) \nonumber\\
&~~~~+ \frac{L_g(1 - \gamma + \bar{\beta}^2)}{2(1 - \gamma)}\cdot\frac{2}{L_g(1 - \bar{\beta}^2)}\left(E_{\hat{\theta}}(x^k, x^{k-1}) - E_{\hat{\theta}}(x^{k+1}, x^k)\right), \nonumber
\end{align*}
where (a) follows from \eqref{upperK}.

Denote $\zeta: = \frac{1 + \bar{\beta}^2 - \gamma}{(1-\gamma)(1 - \bar{\beta}^2)} > 1$ and $A_k := \frac{L_g}{2}(\frac{1}{\gamma}\d^2(x^k, S) - \|x^k - x^{k-1}\|^2)$. Rearranging terms in the above inequality, we have that
\begin{equation*}
(\vartheta + \zeta) \left(E_{\hat{\theta}}(x^{k+1},x^k) - \bar{E}_{\hat{\theta}}^*\right)  \leq A_k - A_{k+1} + \zeta\left(E_{\hat{\theta}}(x^{k},x^{k-1}) - \bar{E}_{\hat{\theta}}^*\right).
\end{equation*}
Dividing $\vartheta + \zeta$ on both sides in the above inequality, we see that
\begin{equation}\label{upperE2}
E_{\hat{\theta}}(x^{k+1},x^k) - \bar{E}_{\hat{\theta}}^* \leq \frac{\zeta}{\vartheta + \zeta}\left(E_{\hat{\theta}}(x^{k},x^{k-1}) - \bar{E}_{\hat{\theta}}^*\right) + \frac{1}{\vartheta + \zeta}A_k - \frac{1}{\vartheta + \zeta}A_{k+1}.
\end{equation}
Since $\vartheta > \frac{1}{2}$ and $\zeta > 1$, we have that
\begin{align}\label{upperA}
&\left|\frac{A_k}{\vartheta + \zeta}\right|\leq \left|A_k\right|\leq \frac{L_g}{2}\left(\frac{1}{\gamma}\d^2(x^k, S) + \|x^k - x^{k-1}\|^2\right) \nonumber\\
&\overset{\rm(a)}\leq\frac{L_g c_0}{2\gamma}\left(E_{\hat{\theta}}(x^k, x^{k-1}) - \bar{E}_{\hat{\theta}}^*\right) + \frac{L_g}{2}\|x^k - x^{k-1}\|^2\nonumber\\
&\overset{\rm(b)}\leq\frac{L_g c_0}{2\gamma}\left(E_{\hat{\theta}}(x^k, x^{k-1}) - \bar{E}_{\hat{\theta}}^*\right) + \frac{1}{(1 - \bar{\beta}^2)}\left(E_{\hat{\theta}}(x^k, x^{k-1}) - E_{\hat{\theta}}(x^{k+1}, x^{k})\right)\nonumber\\
&\overset{\rm(c)} \leq c_1\left(E_{\hat{\theta}}(x^k, x^{k-1}) - \bar{E}_{\hat{\theta}}^*\right),
\end{align}
where (a) holds thanks to \eqref{Eerro}, (b) holds because of \eqref{upperK}, and (c) follows from $E_{\hat{\theta}}(x^{k+1}, x^{k})\geq\bar{E}_{\hat{\theta}}^*$ and $c_1:= \frac{L_g c_0}{2\gamma} + \frac{1}{(1 - \bar{\beta}^2)}$.

Let $\varrho = \frac{c_1 + \frac{\zeta}{\vartheta+\zeta}}{c_1+1}\in(0,1)$. Then one can see that
\begin{equation}\label{divi}
\frac{\zeta}{\vartheta+\zeta} + (1 - \varrho)c_1 = \varrho.
\end{equation}
Then, from \eqref{upperE2}, we obtain that
\begin{align*}
&E_{\hat{\theta}}(x^{k+1},x^k) - \bar{E}_{\hat{\theta}}^* + \frac{1}{\vartheta + \zeta}A_{k+1} \nonumber\\
&\leq \frac{\zeta}{\vartheta + \zeta}\left(E_{\hat{\theta}}(x^{k},x^{k-1}) - \bar{E}_{\hat{\theta}}^*\right) + \frac{1}{\vartheta + \zeta}A_k \nonumber\\
&\overset{\rm(a)} \leq \frac{\zeta}{\vartheta + \zeta}\left(E_{\hat{\theta}}(x^{k},x^{k-1}) - \bar{E}_{\hat{\theta}}^*\right) + \frac{\varrho}{\vartheta + \zeta}A_k + (1 - \varrho)\left|\frac{A_k}{\vartheta + \zeta}\right|\nonumber\\
&\overset{\rm(b)} \leq \left(\frac{\zeta}{\vartheta + \zeta} + (1 - \varrho)c_1\right)\left(E_{\hat{\theta}}(x^{k},x^{k-1}) - \bar{E}_{\hat{\theta}}^*\right) + \frac{\varrho}{\vartheta + \zeta}A_k\nonumber\\
&\overset{\rm(c)} = \varrho \left(E_{\hat{\theta}}(x^{k},x^{k-1}) - \bar{E}_{\hat{\theta}}^* + \frac{1}{\vartheta + \zeta}A_k \right),
\end{align*}
where (a) holds as $\varrho\in(0, 1)$, (b) follows from \eqref{upperA}, and (c) holds because of \eqref{divi}.

Inductively, since $\varrho > 0$, we see that for any $k\geq k_4$,
\begin{equation*}
E_{\hat{\theta}}(x^{k+1},x^k) - \bar{E}_{\hat{\theta}}^* + \frac{1}{\vartheta + \zeta}A_{k+1} \leq \varrho^{k - k_4 +1}\left(E_{\hat{\theta}}(x^{k_4},x^{k_4 - 1}) - \bar{E}_{\hat{\theta}}^* + \frac{1}{\vartheta + \zeta}A_{k_4}\right),
\end{equation*}
which means, there exists $M>0$ such that, for any $k\geq k_4$,
\begin{align}\label{inequ}
0\leq& E_{\hat{\theta}}(x^k,x^{k-1}) - \bar{E}_{\hat{\theta}}^*\leq M\varrho^{k} - \frac{1}{\vartheta + \zeta}A_{k} \notag\\
&\overset{\rm(a)}= M\varrho^{k} - \frac{L_g}{2(\vartheta + \zeta)}\left(\frac{1}{\gamma}\d^2(x^k, S) - \|x^k - x^{k-1}\|^2\right)\notag\\
&\leq M\varrho^{k} + \frac{L_g}{2(\vartheta + \zeta)} \|x^k - x^{k-1}\|^2\notag\\
&\overset{\rm(b)}\leq M\varrho^{k} + \frac{1}{(\vartheta + \zeta)(1 - \bar{\beta}^2)}\left(E_{\hat{\theta}}(x^k,x^{k-1}) - E_{\hat{\theta}}(x^{k+1},x^{k})\right),
\end{align}
where (a) follows from the definition of $A_k$, and (b) holds because of \eqref{upperK}.

Taking $\mu > \max\{\frac{1}{(\vartheta + \zeta)(1 - \bar{\beta}^2)}, \frac{1}{1 - \varrho}\}$. From $\varrho\in(0, 1)$,  we see that
\[
\mu > 1 \text{ and } 1 - \frac{1}{\mu} > \varrho,
\]
and from \eqref{inequ}, we have that
\begin{equation*}
E_{\hat{\theta}}(x^k,x^{k-1}) - \bar{E}_{\hat{\theta}}^* \leq M\varrho^{k} + \mu(E_{\hat{\theta}}(x^k,x^{k-1}) - E_{\hat{\theta}}(x^{k+1},x^{k})),
\end{equation*}
which implies
\begin{equation*}
\mu(E_{\hat{\theta}}(x^{k+1},x^{k}) - \bar{E}_{\hat{\theta}}^*) \leq (\mu - 1)\left(E_{\hat{\theta}}(x^k,x^{k-1}) - \bar{E}_{\hat{\theta}}^*\right) + M\varrho^{k}.
\end{equation*}
Dividing $\mu>0$ on the both sides in the above display, we see that for any $k\geq k_4$,
\begin{align*}
&E_{\hat{\theta}}(x^{k+1},x^{k}) - \bar{E}_{\hat{\theta}}^* \nonumber\\
&\leq \left(1 - \frac{1}{\mu}\right) \left(E_{\hat{\theta}}(x^k,x^{k-1}) - \bar{E}_{\hat{\theta}}^*\right) + \frac{M}{\mu}\varrho^{k} \nonumber\\
&\leq \left(1 - \frac{1}{\mu}\right) \left(E_{\hat{\theta}}(x^k,x^{k-1}) - \bar{E}_{\hat{\theta}}^*\right) + \frac{M}{\mu}\left( \frac{1 - \frac{1}{\mu}}{1 - \frac{1}{\mu} - \varrho} - \frac{\varrho}{1 - \frac{1}{\mu} - \varrho}\right)\varrho^{k} \nonumber\\
&= \left(1 - \frac{1}{\mu}\right) \left(E_{\hat{\theta}}(x^k,x^{k-1}) - \bar{E}_{\hat{\theta}}^* + \frac{M}{\mu(1 - \frac{1}{\mu} - \varrho)}\varrho^{k} \right) - \frac{M}{\mu(1 - \frac{1}{\mu} - \varrho)}\varrho^{k+1}. \nonumber
\end{align*}
Rearranging terms in the above inequality, we have that
\begin{small}
\begin{equation*}
E_{\hat{\theta}}(x^{k+1},x^{k}) - \bar{E}_{\hat{\theta}}^* + \frac{M\varrho^{k+1}}{\mu(1 - \frac{1}{\mu} - \varrho)}
\leq \left(1 - \frac{1}{\mu}\right) \left(E_{\hat{\theta}}(x^k,x^{k-1}) - \bar{E}_{\hat{\theta}}^* + \frac{M\varrho^{k}}{\mu(1 - \frac{1}{\mu} - \varrho)} \right).
\end{equation*}
\end{small}

Inductively, since $1 - \frac{1}{\mu} > \varrho>0$, we see that for any $k\geq k_4$,
\begin{align}\label{inductE}
&E_{\hat{\theta}}(x^{k+1},x^{k}) - \bar{E}_{\hat{\theta}}^*
 \leq
 E_{\hat{\theta}}(x^{k+1},x^{k}) - \bar{E}_{\hat{\theta}}^* + \frac{M\varrho^{k+1}}{\mu(1 - \frac{1}{\mu} - \varrho)} \nonumber\\
& \leq \left(\!1 - \frac{1}{\mu}\!\right)^{k-k_4+1} \!\!\!\ \left(E_{\hat{\theta}}(x^{k_4},x^{k_4-1}) \!-\! \bar{E}_{\hat{\theta}}^* \!+\! \frac{M\varrho^{k_4}}{\mu(1 - \frac{1}{\mu} - \varrho)} \right) \overset{\rm(a)} = c_2 \left(\!1 - \frac{1}{\mu}\!\right)^{k + 1},
\end{align}
where (a) holds with $c_2 := \left(1 - \frac{1}{\mu}\right)^{-k_4}\left(E_{\hat{\theta}}(x^{k_4},x^{k_4-1}) - \bar{E}_{\hat{\theta}}^* + \frac{M}{\mu(1 - \frac{1}{\mu} - \varrho)}\varrho^{k_4} \right)> 0$.

Finally, combining \eqref{upperK} and \eqref{inductE}, we obtain that for any $k\geq k_4$,
\begin{align*}
\|x^k - x^{k - 1}\|^2 & \overset{\rm(a)}\leq \frac{2}{L_g(1 - \bar{\beta})}\left(E_{\hat{\theta}}(x^k, x^{k-1}) - E_{\hat{\theta}}(x^{k+1}, x^k)\right)\\
&\overset{\rm(b)}\leq\frac{2}{L_g(1 - \bar{\beta})} \left(E_{\hat{\theta}}(x^k, x^{k-1}) - \bar{E}_{\hat{\theta}}^*\right)\overset{\rm(c)}\leq \frac{2c_2}{L_g(1 - \bar{\beta})}\left(1 - \frac{1}{\mu}\right)^k,
\end{align*}
where (a) holds because of \eqref{upperK}, (b) follows from $E_{\hat{\theta}}(x^{k+1}, x^k) \geq \bar{E}_{\hat{\theta}}^*$, and (c) holds because of \eqref{inductE}.

Furthermore, for any $k\geq k_4$,
\begin{align*}
\sum_{j = k}^{\infty}\|x^{j+1} - x^j\|&\leq \sum_{j = k}^{\infty} \sqrt{\frac{2c_2}{L_g(1 - \bar{\beta})}}\left(\sqrt{1 - \frac{1}{\mu}}\right)^{j+1}\\
&\leq \sqrt{\frac{2c_2}{L_g(1 - \bar{\beta})}}\cdot\frac{1}{1 - \sqrt{1 - \frac{1}{\mu}}}\left(\sqrt{1 - \frac{1}{\mu}}\right)^{k+1},
\end{align*}
which implies that $\{x^k\}$ is a Cauchy sequence. Combining this with Theorem~\ref{subconver}(ii), by Assumption~\ref{B1} and Assumption~\ref{B2}, we see that $\{x^k\}$ converges to a minimizer $x^*$ of \eqref{eq1}. This completes the proof.
\end{proof}

\section{KL exponent for a concrete model}

In this section, we examine the KL exponent for some structured convex optimization model that arises from applications. Specifically, we consider the following constrained optimization problem:
\begin{equation}\label{KLproblem}
\min_{x\in \R^n}F(x):= P_1(x) + \delta_C(x) + \delta_{g(\cdot) \leq 0}(x),
\end{equation}
where $C$ is a nonempty compact convex set, $P_1:\R^n\rightarrow\R$ is convex, the function $g(x) = (q_1(A_1x), \ldots, q_m(A_m x))$ is continuous with each $q_i: \R^{s_i}\to \R$ being strictly convex and each $A_i\in\R^{s_i\times n}$, and $\{x\in {\rm dom}\,P_1\cap C: g(x)\le 0\}\not=\emptyset$. One can see that, when $q_i$'s are additionally Lipschitz differentiable, problem \eqref{KLproblem} is a special case of \eqref{eq1} with $P_2 = 0$ and $g_i = q_i\circ A_i$ for $i = 1, \ldots, m$.

We first present the definition of the exact penalty parameter of \eqref{KLproblem}.

\begin{definition}[{{\bf Exact penalty parameter}}]\label{Exact}
Let $S_{\eta}:= \Argmin_{x\in C}\{P_1(x) + \eta\max_{i = 1,\cdots,m}[g_i(x)]_+\}$ and $S:= \Argmin_{x\in C\cap\mathcal{F}}\{P_1(x)\} = \Argmin F$. If there exists $\bar{\eta} > 0$ such that for any $\eta \geq \bar{\eta}$, $S_{\eta} = S$, then $\bar{\eta}$ is called the exact penalty parameter of \eqref{KLproblem}.
\end{definition}

Now, we show that if $F$ is KL function with exponent $\alpha$, then
\begin{equation}\label{Feta}
F_{\eta}(x): = P_1(x) + \delta_C(x) + \eta\max_i[g_i(x)]_+
\end{equation}
is KL function with exponent $\alpha$, for any $\eta>\bar{\eta}$, where $\bar{\eta}$ is the exact penalty parameter of \eqref{KLproblem}.

\begin{theorem}\label{FetaKL}
Let $F$ be as in \eqref{KLproblem}, $\bar{x}\in\Argmin F$ and $\bar{\eta}$ be the exact penalty parameter of \eqref{KLproblem}. If $F$ satisfies the KL property with exponent $\alpha$ at $\bar{x}$, then for any $\eta>\bar{\eta}$, $F_{\eta}(x)$ satisfies the KL property with exponent $\alpha$ at $\bar{x}$.
\end{theorem}

\begin{proof}
%In view of \cite[Lemma~2.1]{li18} and the convexity of $F_{\eta}$, it suffices to show that $F_{\eta}$ has a KL property at  $\bar{x}\in\{x : 0\in\partial F_{\eta}(x)\} = \Argmin F_{\eta}$ with exponent $\alpha$.

Since $F$ satisfies the KL property with exponent $\alpha$ at $\bar{x}$, by Lemma~\ref{KLinequ}, there exist $0< a <1$, $c>0$ and $0<\epsilon <1$ such that
\begin{equation}\label{FKL}
\d(x, \Argmin F) \leq c(F(x) - F(\bar{x}))^{\alpha},
\end{equation}
for any $ x\in\dom \partial F = C\cap \mathcal{F}$ satisfying $\|x - \bar{x}\|\leq \epsilon$ %$\d(x, \Argmin F)\leq\epsilon$
and $F(\bar{x}) \leq F(x) \leq F(\bar{x}) + a$, where $\mathcal{F}:= \{x\in\R^n: g_i(x)\leq 0, i = 1,2, \cdots, m\}$.

Since $\bar{\eta}$ is the exact penalty parameter of \eqref{KLproblem}, we see that for any $\eta\geq \bar{\eta}$, $\Argmin F = \Argmin F_{\eta}$ and $\dom\partial F_{\eta} = C$.

From the compactness of $C$ and Assumption~\ref{B2}, by \cite[Corollary~5.14]{Bauschke96}, we have that $\{ C, \mathcal{F}\}$ is boundedly linearly regular, that is, there exists $\kappa >0$, such that for any $x\in C$,
\begin{equation}\label{xdist}
\d(x, C\cap\mathcal{F})\leq \kappa\max(\d(x,C), \d(x, \mathcal{F})) = \kappa\d(x, \mathcal{F}).
\end{equation}

At the same time, we apply Lemma~\ref{RobEB} with $\Omega:=\mathcal{F}$, $g(x) = (g_1(x), g_2(x), \cdots, g_m(x))$, $x^s = \hat{x}$ and $\delta_0: = \left|\max_i{g_i(\hat{x})}\right|$ (where $\hat{x}$ defined in Assumption~\ref{B2}) to obtain that
\begin{equation*}
\d(x, \mathcal{F}) \leq \frac{\|x - \hat{x}\|}{\left|\max_i{g_i(\hat{x})}\right|}\d(0, g(x) + \R^m_+) ~~ \forall x\in C.
\end{equation*}
Since $C$ is compact, we see that there exists $M_1> 0 $, such that
\begin{equation}\label{upperxF}
\d(x, \mathcal{F}) \leq M_1\max_{i = 1,\cdots,m}[g_i(x)]_+ ~~ \forall x\in C.
\end{equation}

Since $P_1:\R^n\to\R$ is convex, without loss of generality, we assume that $P_1$ is locally Lipschitz continuous at $\bar{x}$ with Lipschitz continuity modulus $L_{P_1}$. Hence, there exists $\bar{\epsilon} > 0$, such that,
\begin{equation}\label{LipschP}
|P_1(x) - P_1(\bar{x})|\leq L_{P_1}\|x - \bar{x}\| ~~\forall x\in B(\bar{x}, \bar{\epsilon}).
\end{equation}

Taking $\epsilon_0 := \min\{\epsilon, \bar{\epsilon}, \frac{a}{3L_{P_1}}\}$, then for any $\eta\geq\bar{\eta}$, for any $x\in C = \dom\partial F_{\eta}$ satisfying $\|x - \bar{x}\|\leq \epsilon_0$, we have that
\begin{align}\label{upperproje}
|P_1(\Pi_{C\cap \mathcal{F}}(x)) - P_1(x)| &\leq |P_1(\Pi_{C\cap \mathcal{F}}(x)) - P_1(\bar{x})| + |P_1(\bar{x}) - P_1(x)| \nonumber\\
&\overset{\rm(a)}\leq L_{P_1}\|\Pi_{C\cap \mathcal{F}}(x) - \bar{x}\| + L_{P_1}\|x - \bar{x}\| \leq 2\epsilon_0 L_{P_1},
\end{align}
where $\Pi_{C\cap\mathcal{F}}(x)$ denotes the orthogonal projection of the point $x$ onto $C\cap\mathcal{F}$, (a) follows from \eqref{LipschP} and $\|\Pi_{C\cap\mathcal{F}}(x) - \bar{x}\|\leq\|x - \bar{x}\|\leq \epsilon_0$ (thanks to $\bar{x}\in C\cap\mathcal{F}$ and the projection mapping is nonexpansive).

Letting $a_0 := a - 2\epsilon_0 L_{P_1}>0$. Then for any $\eta\geq\bar{\eta}$, for any $x\in C = \dom\partial F_{\eta}$ satisfying $F_{\eta}(\bar{x}) \leq F_{\eta}(x) \leq F_{\eta}(\bar{x}) + a_0$, we claim that
\begin{equation}\label{projecx}
F(\bar{x}) \leq F(\Pi_{C\cap \mathcal{F}}(x)) \leq F(\bar{x}) + a.
\end{equation}
Indeed, from $\bar{x}\in\Argmin F$, one can see that $F(\bar{x}) \leq F(\Pi_{C\cap \mathcal{F}}(x))$. At the same time, notice that
\[
P_1(x) \leq P_1(x) + \eta\max_i[g_i(x)]_+ = F_{\eta}(x)\leq F_{\eta}(\bar{x}) + a_0 = P_1(\bar{x}) + a - 2\epsilon_0 L_{P_1}.
\]
Combining this with \eqref{upperproje}, one can see that $P_1(\Pi_{C\cap \mathcal{F}}(x))\leq P_1(\bar{x}) + a$. By the definition of $F$, we see that \eqref{projecx} holds.

Hence, for any $\eta>\bar{\eta}$, for any $x\in C = \dom\partial F_{\eta}$ satisfying $\|x - \bar{x}\|\leq \epsilon_0$ and $F_{\eta}(\bar{x}) \leq F_{\eta}(x) \leq F_{\eta}(\bar{x}) + a_0$, we have that
\begin{align}\label{distFeta}
&\d(x, \Argmin F_{\eta})\leq \d(\Pi_{C\cap \mathcal{F}} (x), \Argmin F_{\eta}) + \d(x, C\cap \mathcal{F}) \notag\\
&\overset{\rm(a)} = \d(\Pi_{C\cap \mathcal{F}} (x), \Argmin F) + \d(x, C\cap \mathcal{F}) \notag\\
&\overset{\rm(b)}\leq c(F(\Pi_{C\cap \mathcal{F}} (x)) - F(\bar{x}))^{\alpha} + \kappa\d(x, \mathcal{F}) \notag\\
&\overset{\rm(c)} = c(P_1(\Pi_{C\cap \mathcal{F}} (x)) - P_1(\bar{x}))^{\alpha} + \kappa\d(x, \mathcal{F}) \notag\\
&\overset{\rm(d)}\leq c(P_1(x) - P_1(\bar{x}) + L_{P_1}\d(x, C\cap \mathcal{F}))^{\alpha} + \kappa\d(x, \mathcal{F}) \notag\\
&\overset{\rm(e)}\leq c(P_1(x) - P_1(\bar{x}) + L_{P_1}\kappa\d(x, \mathcal{F}))^{\alpha} + \kappa\d^{\alpha}(x, \mathcal{F}) \notag\\
&\overset{\rm(f)}\leq \bar{c}(P_1(x) - P_1(\bar{x}) + \kappa_1\d(x, \mathcal{F}))^{\alpha} \notag\\
&\overset{\rm(g)}\leq \bar{c}\left(P_1(x) - P_1(\bar{x}) + \kappa_2\max_{i = 1,\cdots,m}[g_i(x)]_+\right)^{\alpha},
\end{align}
where $\Pi_{C\cap\mathcal{F}}(x)$ denotes the orthogonal projection of the point $x$ onto $C\cap\mathcal{F}$, (a) follows from $\Argmin F = \Argmin F_{\eta}$, (b) holds because \eqref{FKL}, \eqref{upperproje} and $\|\Pi_{C\cap\mathcal{F}}(x) - \bar{x}\|\leq\|x - \bar{x}\|\leq\epsilon$ (thanks to $\bar{x}\in C\cap\mathcal{F}$ and the projection mapping is nonexpansive), (c) follows from $\bar{x}\in C\cap\mathcal{F}$, (d) holds because $P_1$ is locally Lipschitz continuous at $\bar{x}$ with Lipschitz constance $L_{P_1}$ (thanks to $P_1$ is convex), (e) follows from \eqref{xdist} and $\d(x,\mathcal{F})\leq \|x - \bar{x}\|\leq\epsilon<1$ and $0\leq \alpha \leq1$, (f) holds because $a^{\alpha} + b^{\alpha}\leq 2^{1 - \alpha}(a + b)^{\alpha}$ for any $a>0$, $b>0$ and $0\leq\alpha\leq1$, $\kappa_1 = L_{P_1}\kappa + \kappa^{\frac{1}{\alpha}}$ and $\bar{c} = 2^{1-\alpha}\max\{c, 1\}$, and (g) follows from \eqref{upperxF} and $\kappa_2 = M_1\kappa_1$.

Now, for any $\eta>\bar{\eta}$, if $\eta\geq\kappa_2$, then, from \eqref{distFeta}, we have that
\[
\d(x, \Argmin F_{\eta})\leq \bar{c}\left(P_1(x) - P_1(\bar{x}) + \eta\max_{i = 1,\cdots,m}[g_i(x)]_+\right)^{\alpha}.
\]
If $\kappa_2 >\eta>\bar{\eta}$, then, from \eqref{distFeta}, we obtain that
\begin{align*}
&\d(x, \Argmin F_{\eta})\leq \bar{c}\left(P_1(x) \!-\! P_1(\bar{x}) \!+\! \bar{\eta}\max_{i = 1,\cdots,m}[g_i(x)]_+ \!+\! (\kappa_2 - \bar{\eta})\max_{i = 1,\cdots,m}[g_i(x)]_+\right)^{\alpha}\\
&\overset{\rm(a)} \leq \bar{c}\left(\frac{\kappa_2 - \bar{\eta}}{\eta - \bar{\eta}}\right)^{\alpha}\left(P_1(x) - P_1(\bar{x}) + \bar{\eta}\max_{i = 1,\cdots,m}[g_i(x)]_+ + (\eta - \bar{\eta}) \max_{i = 1,\cdots,m}[g_i(x)]_+\right)^{\alpha}\\
&= \bar{c}\left(\frac{\kappa_2 - \bar{\eta}}{\eta - \bar{\eta}}\right)^{\alpha}\left(F_{\eta}(x) - F_{\eta}(\bar{x})\right)^{\alpha},
\end{align*}
where (a) holds because $a + b\leq\frac{1}{\epsilon}(a + \epsilon b)$ for any $a\geq 0$, $b \geq 0$ and $0<\epsilon\leq 1$ (let $a:= P_1(x) - P_1(\bar{x}) + \bar{\eta}\max_{i = 1,\cdots,m}[g_i(x)]_+ \geq 0$ (thanks to $\bar{x}\in\Argmin F$, $x\in C$ and the definition of $\bar{\eta}$), $b:= (\eta - \bar{\eta})\max_{i = 1,\cdots,m} [g_i(x)]_+ \geq 0$, $\epsilon: = \frac{\eta - \bar{\eta}}{\kappa_2 - \bar{\eta}}\in(0,1)$). This completes the proof.
\end{proof}

From \cite[Thoerem~5.1]{zhang23}, using Theorem~\ref{FetaKL}, we have that under some suitable conditions, $F_{\eta}$ satisfies the KL property, for any $\eta>\bar{\eta}$, where $\bar{\eta}$ is the exact penalty parameter of \eqref{KLproblem}.

\begin{corollary}
Let $F$ be as in \eqref{KLproblem}, $\bar{x}\in\Argmin F$ and $\bar{\eta}$ be the exact penalty parameter of \eqref{KLproblem}. Suppose the following conditions hold:
\begin{enumerate}[{\rm (i)}]
  \item There exists a Lagrange multiplier $\bar{\lambda}\in\R^m_+$ for \eqref{KLproblem} and $x\mapsto P_1(x) + \delta_C(x) + \langle\bar{\lambda}, g(x)\rangle$ is a KL function with exponent $\alpha\in(0,1)$.
  \item The strict complementarity condition holds at $(\bar{x}, \bar{\lambda})$, i.e., for any $i$ satisfying $\bar{\lambda}_i = 0$, it holds that $q_i(A_i\bar{x}) < 0$.
\end{enumerate}
Then for any $\eta>\bar{\eta}$, $F_{\eta}$ satisfies the KL property with exponent $\alpha$ at $\bar{x}$.
\end{corollary}



The next corollary deals with \eqref{KLproblem} with $m = 1$ in \cite[Corollary~5.1]{zhang23}.

\begin{corollary}\label{CorrKL}
Let $F$ be as in \eqref{KLproblem} with $m = 1$, and $\bar{\eta}$ be the exact penalty parameter of \eqref{KLproblem}. Suppose the following conditions hold:
\begin{enumerate}[{\rm (i)}]
  \item $\inf_{x\in \R^n} P_1(x) + \delta_C(x) < \inf_{x\in \R^n} \{P_1(x) + \delta_C(x): q_1(A_1x)\leq 0\}$.
  \item There exists a Lagrange multiplier $\bar{\lambda}\in\R_+$ and $x\mapsto P_1(x) + \delta_C(x) + \bar{\lambda} q_1(A_1x)$ is a KL function with exponent $\alpha\in(0,1)$.
\end{enumerate}
Then, for any $\eta>\bar{\eta}$, $F_{\eta}$ is KL function with exponent $\alpha$.
\end{corollary}

Moreover, if $m = 1$ in \eqref{KLproblem}, then Assumption~\ref{B2} become the following assumption.
\begin{assumption}\label{B3}
For \eqref{KLproblem} with $m = 1$, there exists $\hat{x}\in C$ with $q_1(A_1\hat{x}) < 0$.
\end{assumption}

\begin{remark}\label{RemarkKL}
Suppose that $P_1 = \|\cdot\|_1$ in \eqref{KLproblem} and $\bar{\eta}$ be the exact penalty parameter of \eqref{KLproblem}. We deduce from \cite[Corollary~5.1]{li18}, Corollary~\ref{CorrKL} and Theorem~\ref{FetaKL} that, for any $\eta>\bar{\eta}$, the KL exponent of the corresponding $F_{\eta}$ is $\frac{1}{2}$ if $m = 1$, $q_1:\mathbb{R}^{s_1} \rightarrow \mathbb{R}$ takes one of the following forms with $b\in \R^{s_1}$ and $\sigma > 0$ chosen so that the origin is not feasible and that Assumption~\ref{B3} holds:
\begin{enumerate}[{\rm (i)}]
   \item (Basis pursuit denoising \cite{Ca18}) $q_1(z) = \frac{1}{2}\|z - b\|^2 - \sigma$.
   \item (Logistic loss \cite{HoLS13}) $q_1(z) = \sum_{i=1}^{s_1}\log(1 + \exp(b_iz_i)) - \sigma$ for some $b\in \{-1,1\}^{s_1}$.
   \item (Poisson loss \cite{zo04}) $q_1(z) = \sum_{i=1}^{s_1}(-b_iz_i + \exp(z_i)) - \sigma$.
 \end{enumerate}
\end{remark}




\section{Applications in compressed sensing}

In this section, we consider the following model:

\begin{equation}\label{CompSen}
\begin{aligned}
\min_{x\in\R^n}\quad &\|x\|_1 - \mu\|x\| \\
{\rm s.t.} \quad & h(Ax - b)\leq \sigma,
\end{aligned}
\end{equation}
where $\mu\in[0, 1]$, $A\in\R^{q\times n}$ has full row rank, $b\in \R^q$, $h: \R^q\rightarrow\R_+$ is an analytic function whose gradient is Lipschitz continuous with modulus $L_{h}$ and satisfies $h(0) = 0$, and $\sigma\in (0, h(-b))$.

Although the feasible region of \eqref{CompSen} is unbounded and Algorithm~\ref{alg:Framwork} cannot be directly applied to solving \eqref{CompSen}, one can argue as in the discussion following \cite[(6.2)]{zhang23} that \eqref{CompSen} is equivalent to the following model:
\begin{equation}\label{CompSen1}
\begin{aligned}
\min_{x\in\R^n}\quad &\|x\|_1 - \mu\|x\| \\
{\rm s.t.} \quad & h(Ax - b) \leq \sigma,\\
           \quad & \|x\|_{\infty} \leq M,
\end{aligned}
\end{equation}
where $M: = (1 - \mu)^{-1}\Big(\|\tilde{x}\|_1 - \mu\|\tilde{x}\|\Big)$, for some feasible point $\tilde{x}$.

Problem \eqref{CompSen1} is a special case of \eqref{eq1} with $P_1: = \|x\|_1$, $P_2(x) := \mu\|x\|$, $g(x) := h(Ax - b)$ and $C: = \{x: \|x\|_{\infty}\leq M\}$. Since $A$ has full row rank and $h(0) = 0< \sigma$, we see that $\{x: g(x) < 0 \}\not=\emptyset$.

Now, we consider two specific choices of $h$.

\subsection{$h(\cdot) = \frac{1}{2}\|\cdot\|^2$}
In this subsection, we take $h(\cdot) = \frac{1}{2}\|\cdot\|^2$, then \eqref{CompSen1} becomes
\begin{equation}\label{CompSen1-1}
\begin{aligned}
\min_{x\in\R^n}\quad &\|x\|_1 - \mu\|x\| \\
{\rm s.t.} \quad & \frac{1}{2}\|Ax - b\|^2 \leq \sigma,\\
           \quad & \|x\|_{\infty} \leq M,
\end{aligned}
\end{equation}
where $M: = (1 - \mu)^{-1}\Big(\|A^{\dag}b\|_1 - \mu\|A^{\dag}b\|\Big)$.

It is easy to see that $h$ is convex and the Slater condition holds for the feasible region of \eqref{CompSen1-1}. Then the Assumption~\ref{A1} holds. One can apply Theorem~\ref{subconver}(ii) with $\ell = 0$ to deducing the convergence of the sequence generated by Algorithm~\ref{alg:Framwork} and apply Theorem~\ref{th2.2} to obtaining the convergence rate of the sequence generated by Algorithm~\ref{alg:Framwork} when applied to solving \eqref{CompSen1-1}. When $\mu = 0$ in \eqref{CompSen1-1}, since $\sigma\in(0, \frac{1}{2}\|b\|^2)$ and $A$ has full row rank, we see from Corollary~\ref{RemarkKL} that $x\rightarrow \|x\|_1 + \delta_C(x) + \eta\max_{i = 1,\cdots,m} [g_i(x)]_+$ is a KL function with exponent $\frac{1}{2}$. Therefore, the sequence $\{x^k\}$ generated by Algorithm~\ref{alg:Framwork} for \eqref{CompSen1-1} converges locally linearly.

We compare SCP$_{\rm ls}$ in \cite{yu21}, ESQM(taking $\beta_k \equiv 0$ in Algorithm~\ref{alg:Framwork})with ESQM$_e$ in Algorithm~\ref{alg:Framwork}. We use the same parameter settings for SCP$_{\rm ls}$ in \cite{yu21}, and the initial point of SCP$_{\rm ls}$ is chosen as $x^0 = A^\dagger b$. In ESQM and ESQM$_e$, we take $L_g = \|A\|^2$, $\ell_g = 0$, $d = 1$, $\theta_0 = 1$, the initial point are chosen as $x^0 = 0$. We terminate all algorithms when $\|x^{k+1} - x^k\|< 10^{-8}\max\{1, \|x^{k+1}\|\}$.

We use FISTA (which reset parameters with fixed and adaptive restart) to generate the sequence $\{\beta_k\}$. In more detail, setting the initial values $\vartheta_{-1}=\vartheta_{0}=1$, for $k\geq 0$, defining that
\begin{align}\label{beta}
\beta_k = \frac{\vartheta_{k-1}-1}{\vartheta_{k}},
\end{align}
where $\vartheta_{k+1}=\frac{1+\sqrt{1+4\vartheta_{k}^2}}{2}$. The reset process is as follows: we fix a positive number $K:=200$, and reset parameters $\vartheta_{k-1}=\vartheta_{k}=1$ every $K$ iterations under appropriate conditions, while the adaptive restart scheme amounts to resetting $\vartheta_{k-1}=\vartheta_{k}=1$ whenever $\langle y^{k-1} - x^k, x^k - x^{k-1} \rangle>0$. By this method, $\{\beta_k\}$ satisfies $\{\beta_k\}\subseteq[0,1)$ and $\sup_{k}\beta_k<1$.

We generate an $A\in\R^{q\times n}$ with independent and identically distributed (i.i.d.) standard Gaussian entries, and then normalize this matrix so that each column of $A$ has unit norm. Then we choose a subset $T$ of size $k$ uniformly at random from $\{1, 2, \cdots m\}$ and an $k-$sparse vector $x_{\rm orig}$ having i.i.d. standard Gaussian entries is generated. We let $b = Ax_{\rm orig} + 0.01\cdot\hat{n}$ with $\hat{n}$ being a random vector with i.i.d. standard Gaussian entries and $\sigma = \frac{1}{2}\sigma_1^2$ with $\sigma_1 = 1.1\cdot\|0.01\cdot\hat{n}\|$.

In our numerical tests, we let $\mu =0.95$, $(p,n,k) = (720i,2560i,80i)$ with $i\in \{2, 4, 6, 8, 10\}$. For each $i$, we generate 20 random data as described above. We present the computational results in Table~\ref{table1}, averaged over the $20$ random instances. Here, we present the time for computing the QR decomposition of $A^T$ (denoted by $t_{\rm QR}$), the time for computing $\|A\|$ (denoted by $t_{\|A\|}$),  the time for computing $x^0 = A^\dagger b$ given the QR factorization of $A^T$ (denoted by $t_{A^\dagger b}$),\footnote{For SCP$_{\rm ls}$, we taking the initial point $x^0 = A^\dagger b$. For ESQM and ESQM$_{\rm e}$, we taking the initial point $x^0 = 0$.} the CPU times of the algorithms (CPU time), the number of iterations (denoted by Iter), the recovery errors $\text{RecErr} := \frac{\|x^* - \xorig\|}{\max\{1, \|\xorig\|\}}$ and the residuals ${\rm Residual} := \frac{\|Ax^* - b\| - \sigma }{\sigma}$, where $x^*$ is the approximate solution returned by the respective algorithm.


\begin{table}[h]
{\color{black}
\caption{Computational results for problem \eqref{CompSen1-1}}\label{table1}
\begin{center}
{\footnotesize
\begin{tabular}{|c|c|c|c|c|c|c|}\hline
\phantom{\diagbox{Date}{$i$}} & \multicolumn{1}{|c|}{Method} & \multicolumn{1}{|c|}{$i = 2$} & \multicolumn{1}{c|}{ $i = 4$ }
& \multicolumn{1}{c|}{ $i = 6$ } & \multicolumn{1}{|c|}{ $i = 8$ } & \multicolumn{1}{c|}{ $i = 10$ }
\\\cline{1-7}\multirow{6}*{CPU time} & \multirow{1}*{$t_{\rm QR}$}
&  0.727 &  4.977 & 17.050 & 41.958 & 97.176        \\\cline{2-2} \multirow{1}*{} & \multirow{1}*{$t_{A^\dagger b}$}
&  0.007 &  0.030 &  0.069 &  0.123 &  0.252        \\\cline{2-2} \multirow{1}*{} & \multirow{1}*{$t_{\|A\|}$}
&  0.734 &  1.693 &  5.871 & 13.707 & 30.315        \\\cline{2-7} \multirow{1}*{} & \multirow{1}*{SCP$_{\rm ls}$}
&  1.784 &  5.817 & 12.781 & 21.565 & 44.421        \\\cline{2-2} \multirow{1}*{}  & \multirow{1}*{ESQM}
&  5.552 & 20.651 & 47.041 & 79.073 & 161.768        \\\cline{2-2} \multirow{1}*{}  & \multirow{1}*{ESQM$_{{\rm e}}$}
&  0.759 &  2.715 &  6.345 & 10.671 & 22.689        \\\cline{1-7} \multirow{3}*{Iter} & \multirow{1}*{SCP$_{\rm ls}$}
&    138 &    140 &    137 &    137 &    140        \\\cline{2-2} \multirow{1}*{}     & \multirow{1}*{ESQM}
&    753 &    819 &    805 &    806 &    804        \\\cline{2-2} \multirow{1}*{}     & \multirow{1}*{ESQM$_{{\rm e}}$}
&    112 &    112 &    114 &    113 &    114        \\\cline{1-7} \multirow{3}*{RecErr} & \multirow{1}*{SCP$_{\rm ls}$}
&  0.017 &  0.017 &  0.017 &  0.017 &  0.018       \\\cline{2-2} \multirow{1}*{} & \multirow{1}*{ESQM}
&  0.017 &  0.017 &  0.017 &  0.017 &  0.018       \\\cline{2-2} \multirow{1}*{} & \multirow{1}*{ESQM$_{{\rm e}}$}
&  0.017 &  0.017 &  0.017 &  0.017 &  0.018       \\\cline{1-7} \multirow{3}*{Residual} & \multirow{1}*{SCP$_{\rm ls}$}
& -9.28e-14 & -3.25e-13 & -2.31e-13 & -2.69e-13 & -2.15e-13        \\\cline{2-2} \multirow{1}*{} & \multirow{1}*{ESQM}
& -3.52e-09 & -1.72e-09 & -1.14e-09 & -8.59e-10 & -6.76e-10        \\\cline{2-2} \multirow{1}*{} & \multirow{1}*{ESQM$_{{\rm e}}$}
& -8.26e-10 & -1.18e-09 & -6.44e-10 & -7.86e-10 & -4.78e-10        \\\cline{1-7}
\end{tabular}
}
\end{center}
}
\end{table}

From Table~\ref{table1}, one can see that ESQM$_{{\rm e}}$ is the fastest in terms of both time and iteration steps, and the recovery errors of SCP$_{\rm ls}$, ESQM and ESQM$_{{\rm e}}$ are comparable.

\subsection{ When $h$ is the Lorentzian norm. }

In this subsection, we consider $h$ is the Lorentzian norm, which is defined as following: for any given $\gamma>0$, the Lorentzian norm of a vector $y\in\R^q$ is defined as
\[
\|y\|_{L L_2, \gamma}:=\sum_{i=1}^q \log\left(1 + \frac{y_i^2}{\gamma^2}\right).
\]
Then, problem~\eqref{CompSen1} becomes the following problem:
\begin{equation}\label{CompSen1-2}
\begin{aligned}
\min_{x\in\R^n}\quad & \|x\|_1 - \mu\|x\| \\
{\rm s.t.}\quad & \|Ax - b\|_{L L_2, \gamma}\leq \sigma,\\
\quad & \|x\|_{\infty} < M,
\end{aligned}
\end{equation}
where $M: = (1 - \mu)^{-1}\Big(\|A^{\dag}b\|_1 - \mu\|A^{\dag}b\|\Big)$.

One can see that the mapping $g(y): = \|y\|_{L L_2, \gamma} - \sigma$ has Lipschitz continuous gradient with modulus $\frac{2}{\gamma^2}$. The following proposition shows that $g$ can be represent the difference of two convex functions $g_1$ and $g_2$ with Lipschitz continuous gradients, and the continuity modulus of Lipschitz continuous gradients of $g_1$ is $\frac{2}{\gamma^2}$.

\begin{proposition}
Let $g(y): = \|y\|_{L L_2, \gamma} - \sigma$. Then there exist two convex functions $g_1$ and $g_2$ with Lipschitz continuous gradients, such that, $g(y) = g_1(y) - g_2(y)$ and the continuity modulus of Lipschitz continuous gradients of $g_1$ is $\frac{2}{\gamma^2}$.
\end{proposition}

\begin{proof}
For any $t\in \R$,
\[
\left(\log(1 + t^2)\right)^{''} = \frac{2(1 - t^2)}{(1 + t^2)^2} = \left[\frac{2(1 - t^2)}{(1+t^2)^2}\right]_+ - \left[\frac{2(1 - t^2)}{(1+t^2)^2}\right]_-.
\]

Letting
\[
r_1(t) =\int_0^t (t-s)\left[\frac{2(1 - s^2)}{(1+s^2)^2}\right]_+ds{\text{~ and ~}} r_2(t) =\int_0^t (t-s)\left[\frac{2(1 - s^2)}{(1+s^2)^2}\right]_-ds.
\]
Then, we have that
\[
r_1^{''}(t) = \left[\frac{2(1 - t^2)}{(1 + t^2)^2}\right]_+ {\text{~ and  ~}}  r_2^{''}(t) = \left[\frac{2(1 - t^2)}{(1 + t^2)^2}\right]_-.
\]

Taking
\[
g_1(y) = \sum_{i = 1}^m r_1(y_i) - \sigma, \text{~  and  ~} g_2(y) = \sum_{i = 1}^m r_2(y_i),
\]
one can see that $g_1$ and $g_2$ are two convex functions with Lipschitz continuous gradients, and $g(y) = g_1(y) - g_2(y)$. Furthermore, the continuity modulus of Lipschitz continuous gradients of $g_1$ and $g_2$ are $\frac{2}{\gamma^2}$ and $\frac{1}{4\gamma^2}$, respectively. This completes the proof.
\end{proof}



If Assumption~\ref{A1} holds, one can apply Theorem~\ref{subconver}(ii) with $L_g = \frac{2\|A\|}{\gamma^2}$ and $\ell_g = \frac{\|A\|}{4\gamma^2}$ to deducing the convergence of the sequence generated by Algorithm~\ref{alg:Framwork} and apply Theorem~\ref{th2.2} to obtaining the convergence rate of the sequence generated by Algorithm~\ref{alg:Framwork} when applied to solving \eqref{CompSen1-2}.



We compare SCP$_{\rm ls}$ in \cite{yu21}, ESQM(taking $\beta_k \equiv 0$ in Algorithm~\ref{alg:Framwork})with ESQM$_e$ in Algorithm~\ref{alg:Framwork}. We use the same parameter settings for SCP$_{\rm ls}$ in \cite{yu21}, and the initial point of SCP$_{\rm ls}$ is chosen as $x^0 = A^\dagger b$. In ESQM and ESQM$_e$, we take $L_g = \frac{2\|A\|}{\gamma^2}$, $\ell_g = \frac{\|A\|}{4\gamma^2}$, $d = \frac{\gamma^2}{20}$, $\theta_0 = 0.05$, the initial point are chosen as $x^0 = 0$. We terminate all algorithms when $\|x^{k+1} - x^k\|< 10^{-8}\max\{1, \|x^{k+1}\|\}$.

We use FISTA (which reset parameters with fixed and adaptive restart) to generate the sequence $\{\beta_k\}$. In more detail, setting the initial values $\vartheta_{-1}=\vartheta_{0}=1$, for $k\geq 0$, defining that
\begin{align}\label{beta}
\beta_k = \frac{\vartheta_{k-1}-1}{\vartheta_{k}},
\end{align}
where $\vartheta_{k+1}=\frac{1+\sqrt{1+4\vartheta_{k}^2}}{2}$. The reset process is as follows: we fix a positive number $K = 49$,\footnote{In this case, from the definition of $\beta_k$, we have that $\beta_k$ is increase and $\beta_{50} = 0.9428 = \frac{2}{2.25} = \sqrt{\frac{L_g}{L_g + \ell_g}}$. Therefore, we reset parameters $\vartheta_{k-1}=\vartheta_{k}=1$ every 49 iterations, in order to ensure $\bar{\beta}: = \sup_{k}\beta_k<\sqrt{\frac{L_g}{L_g + \ell_g}}$.} and reset parameters $\vartheta_{k-1}=\vartheta_{k}=1$ every $K$ iterations under appropriate conditions, while the adaptive restart scheme amounts to resetting $\vartheta_{k-1}=\vartheta_{k}=1$ whenever $\langle y^{k-1} - x^k, x^k - x^{k-1} \rangle>0$. By this method, $\{\beta_k\}$ satisfies $\{\beta_k\}\subseteq\left[0,\sqrt{\frac{L_g}{L_g + \ell_g}}\right)$ and $ \sup_{k}\beta_k<\sqrt{\frac{L_g}{L_g + \ell_g}}$.


We generate an $A\in\R^{q\times n}$ with independent and identically distributed standard Gaussian entries, and then normalize this matrix so that each column of $A$ has unit norm. Then we choose a subset $T$ of size $k$ uniformly at random from $\{1, 2, \cdots m\}$ and an $k-$sparse vector $x_{\rm orig}$ having i.i.d. standard Gaussian entries is generated. We let $b = Ax_{\rm orig} + 0.01\cdot\bar{n}$ with $\bar{n}_i\sim{\rm Cauchy}(0, 1)$, that is $\bar{n}_i := \tan(\pi(\tilde{n}_i - \frac{1}{2}))$ with $\tilde{n}$ being a random vector with i.i.d. entries uniformly chosen in $[0, 1]$ and $\sigma = 1.1\cdot\|0.01\cdot\tilde{n}\|_{L L_2, \gamma}$ with $\gamma = 0.05$.


In our numerical tests, we let $\mu =0.95$, $(p,n,k) = (720i,2560i,80i)$ with $i\in \{2, 4, 6, 8, 10\}$. For each $i$, we generate 20 random data as described above. We present the computational results in Table~\ref{table2}, averaged over the $20$ random instances. Here, we present the time for computing the QR decomposition of $A^T$ (denoted by $t_{\rm QR}$), the time for computing $\|A\|$ (denoted by $t_{\|A\|}$), the time for computing $x^0 = A^\dagger b$ given the QR factorization of $A^T$ (denoted by $t_{A^\dagger b}$),\footnote{For SCP$_{\rm ls}$, we taking the initial point $x^0 = A^\dagger b$. For ESQM and ESQM$_{\rm e}$, we taking the initial point $x^0 = 0$.} the CPU times of the algorithms (CPU time), the number of iterations (denoted by Iter), the recovery errors $\text{RecErr} := \frac{\|x^* - \xorig\|}{\max\{1, \|\xorig\|\}}$ and the residuals ${\rm Residual} := \frac{\|Ax^* - b\| - \sigma }{\sigma}$, where $x^*$ is the approximate solution returned by the respective algorithm.




\begin{table}[h]
{\color{black}
\caption{Computational results for problem \eqref{CompSen1-2}}\label{table2}
\begin{center}
{\footnotesize
\begin{tabular}{|c|c|c|c|c|c|c|}\hline
\phantom{\diagbox{Date}{$i$}} & \multicolumn{1}{|c|}{Method} & \multicolumn{1}{|c|}{$i = 2$} & \multicolumn{1}{c|}{ $i = 4$ }
& \multicolumn{1}{c|}{ $i = 6$ } & \multicolumn{1}{|c|}{ $i = 8$ } & \multicolumn{1}{c|}{ $i = 10$ }
\\
\cline{1-7}\multirow{6}*{CPU time} & \multirow{1}*{$t_{\rm QR}$}
&  0.689 &  7.391 & 56.262 & 129.634 & 209.607      \\
\cline{2-2} \multirow{1}*{} & \multirow{1}*{$t_{A^\dagger b}$}
&  0.008 &  0.048 &  0.237 &  0.424 &  0.578     \\
\cline{2-2} \multirow{1}*{} & \multirow{1}*{$t_{\|A\|}$}
&  0.721 &  2.571 & 18.111 & 39.223 & 63.022     \\
\cline{2-7} \multirow{1}*{} & \multirow{1}*{SCP$_{\rm ls}$}
&  2.232 & 19.284 & 44.192 & 220.783 & 122.567       \\
\cline{2-2} \multirow{1}*{}  & \multirow{1}*{ESQM}
&  5.639 & 32.396 & 149.598 & 265.260 & 354.146        \\
\cline{2-2} \multirow{1}*{}  & \multirow{1}*{ESQM$_{{\rm e}}$}
&  0.997 &  5.705 & 24.665 & 44.876 & 58.754     \\
\cline{1-7} \multirow{3}*{Iter} & \multirow{1}*{SCP$_{\rm ls}$}
&    207 &    332 &    196 &    596 &    266     \\
\cline{2-2} \multirow{1}*{}     & \multirow{1}*{ESQM}
&   1107 &   1122 &   1125 &   1151 &   1177     \\
\cline{2-2} \multirow{1}*{}     & \multirow{1}*{ESQM$_{{\rm e}}$}
&    189 &    191 &    185 &    194 &    196     \\
\cline{1-7} \multirow{3}*{RecErr} & \multirow{1}*{SCP$_{\rm ls}$}
&  0.082 &  0.086 &  0.086 &  0.086 &  0.088  \\
\cline{2-2} \multirow{1}*{} & \multirow{1}*{ESQM}
&  0.082 &  0.086 &  0.086 &  0.086 &  0.088  \\
\cline{2-2} \multirow{1}*{} & \multirow{1}*{ESQM$_{{\rm e}}$}
&  0.082 &  0.086 &  0.086 &  0.086 &  0.088  \\
\cline{1-7} \multirow{3}*{Residual} & \multirow{1}*{SCP$_{\rm ls}$}
& -5.53e-15 & -6.57e-15 & -6.76e-15 & -6.08e-15 & -7.00e-15       \\
\cline{2-2} \multirow{1}*{} & \multirow{1}*{ESQM}
& -5.96e-14 & -2.21e-15 & -1.44e-14 & -4.53e-15 & -1.37e-15       \\
\cline{2-2} \multirow{1}*{} & \multirow{1}*{ESQM$_{{\rm e}}$}
& -2.73e-15 & 2.18e-15 & 1.35e-14 & 2.79e-15 & 1.31e-14   \\
\cline{1-7}
\end{tabular}
}
\end{center}
}
\end{table}

From Table~\ref{table2}, one can see that ESQM$_{{\rm e}}$ is the fastest in terms of both time and iteration steps, and the recovery errors of SCP$_{\rm ls}$, ESQM and ESQM$_{{\rm e}}$ are comparable.




\bmhead{Acknowledgments}
The second author is supported in part by the Hong Kong Research Grants Council PolyU153001/22p.





\begin{thebibliography}{99}
\bibitem{attouch10}
H. Attouch, J. Bolte, and P. Redont.
\newblock Proximal alternating minimization and projection methods for nonconvex problems: An approach based on the Kurdyka-{\L}ojasiewicz inequality.
\newblock {\em Mathematics of Operations Research} 35, 438--457, 2010.


\bibitem{attouch13}
H. Attouch, J. Bolte, and B. F. Svaiter.
\newblock Convergence of descent methods for semi-algebraic and tame problems: proximal algorithms, forward--backward splitting, and regularized Gauss--Seidel methods.
\newblock {\em Mathematical Programming} 137, 91--129, 2013.

\bibitem{Ausleder13}
A. Auslender.
\newblock An Extended Sequential Quadratically Constrained Quadratic Programming Algorithm for Nonlinear, Semidefinite, and Second-Order Cone Programming.
\newblock{\em Journal of Optimization Theory and Applications} 156, 183--212, 2013.


\bibitem{Bauschke96}
H. H. Bauschke, and J. M. Borwein.
\newblock On Projection Algorithms for Solving Convex Feasibility Problems.
\newblock {\em SIAM Review} 38(3), 367--426, 1996.


\bibitem{beck09}
A. Beck and M. Teboulle.
\newblock Fast gradient-based algorithms for constrained total variation image denoising and deblurring problems.
\newblock {\em IEEE Transactions on Image Processing} 18, 2419--2434, 2009.


\bibitem{bolte07}
J. Bolte, A. Daniilidis and A. Lewis.
\newblock The {\L}ojasiewicz inequality for nonsmooth subanalytic functions with applications to subgradient dynamical systems.
\newblock {\em SIAM Journal on Optimization} 17, 1205--1223, 2007.


%\bibitem{bolte16}
%J. Bolte and E. Pauwels.
%\newblock Majorization-minimization procedures and convergence of SQP methods for semi-algebraic and tame programs.
%\newblock {\em Mathematics of Operations Research} 41, 442--465, 2016.


\bibitem{bolte14}
J. Bolte, S. Sabach and M. Teboulle.
\newblock Proximal alternating linearized minimization for nonconvex and nonsmooth problems.
\newblock {\em Mathematical Programming} 146, 459--494, 2014.

%\bibitem{bolte17}
%J. Bolte, T.P. Nguyen, J. Peypouquet, B. W. Suter.
%\newblock From error bounds to the complexity of first-order descent methods for convex functions.
%\newblock {\em Mathematical Programming} 165, 471�C507, 2017.

\bibitem{Ca18}
E. J. Cand\'{e}s.
\newblock The restricted isometry property and its implications for compressed sensing.
\newblock {\em  Comptes Rendus Mathematique} 346, 589--592, 2008.

\bibitem{HoLS13}
J. D. W. Hosmer, S. Lemeshow and R. X. Sturdivant.
\newblock {\em  Applied Logistic Regression}.
\newblock John Wiley Sons, 3rd edition, 2013.

\bibitem{li18}
G. Y. Li and T. K. Pong.
\newblock Calculus of the exponent of Kurdyka--{\L}ojasiewicz inequality and its applications to linear convergence of first-order methods.
\newblock {\em Foundations of Computational Mathematics} 18, 1199--1232, 2018.

%\bibitem{liu19}
%T. X. Liu, T. K. Pong, and A. Takeda.
%\newblock A refined convergence analysis of pDCA$_e$ with applications to simultaneous sparse recovery and outlier detection.
%\newblock {\em Computational Optimization and Applications} 73, 69--100, 2019.

%\bibitem{nesterov83}
%Y. Nesterov.
%\newblock A method for unconstrained convex minimization problem with the rate of convergence O ($\frac{1}{k^2}$).
%\newblock {\em Doklady AN USSR} 269, 543--547, 1983.
%
%\bibitem{Tseng08}
%P. Tseng.
%\newblock On accelerated proximal gradient methods for convex-concave optimization[J].
%\newblock {\em submitted to SIAM Journal on Optimization} 2, 2008.

%\bibitem{Tseng10}
%P. Tseng.
%\newblock Approximation accuracy, gradient methods, and error bound for structured convex optimization[J].
%\newblock {\em Mathematical Programming} 125, 263-295, 2010.

\bibitem{Ro70}
R. T. Rockafellar.
\newblock {\em Convex Analysis.}
\newblock Princeton University Press, Princeton, 1970.

\bibitem{rock97a}
R. T. Rockafellar and R. J-B. Wets.
\newblock {\em Variational Analysis.}
\newblock Springer, 1997.

\bibitem{Rob75}
S. M. Robinson.
\newblock An application of error bounds for convex programming in a linear space.
\newblock {\em SIAM Journal on Control} 13, 271--273, 1975.

\bibitem{Tu98}
H. Tuy.
\newblock {\em Convex Analysis and Global Optimization}.
\newblock Springer, 1998.

\bibitem{wen17}
B. Wen, X. J. Chen, and T. K. Pong.
\newblock Linear convergence of proximal gradient algorithm with extrapolation for a class of nonconvex nonsmooth minimization problems.
\newblock {\em SIAM Journal on Optimization} 27, 124�C145, 2017.

\bibitem{wen18}
B. Wen, X. J. Chen, and T. K. Pong.
\newblock A proximal difference-of-convex algorithm with extrapolation.
\newblock {\em Computational Optimization and Applications} 69, 297--324, 2018.


\bibitem{yu21}
P. R. Yu, T. K. Pong and Z. S. Lu.
\newblock Convergence rate analysis of a sequential convex programming method with line search for a class of constrained difference-of-convex optimization problems.
\newblock {\em SIAM Journal on Optimization} 31, 2024--2054, 2021.

\bibitem{zhang23}
Y. L. Zhang, G.Y. Li, T. K. Pong and S. Q. Xu.
\newblock Retraction-based first-order feasible methods for difference-of-convex programs with smooth inequality and simple geometric constraints.
\newblock {\em Advances in Computational Mathematics} 49, Article number: 8, 2023.

\bibitem{zo04}
G. Zou.
\newblock A modified Poisson regression approach to prospective studies with binary data.
\newblock {\em American Journal of Epidemiology} 159, 702--706, 2004.




\end{thebibliography}


\end{document}
